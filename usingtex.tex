% -*- coding: utf-8 -*-
% This is part of the book TeX for the Impatient.
% Copyright (C) 2003 Paul W. Abrahams, Kathryn A. Hargreaves, Karl Berry.
% See file fdl.tex for copying conditions.

\input macros
\chapter{使用 \TeX}

\chapterdef{usingtex}

% Avoid underfull box complaint about the empty paragraph
% that precedes the first section heading.
%
\def\par{{\parfillskip = 0pt plus 1fil\endgraf}\let\par=\endgraf}
\vglue-\abovesectionskip % we've skipped enough already
\vskip0pt % Make \combineskips work.

\section 从键盘输入变到油墨

\subsection 你所需要的程序和文件

为了制作一个 \TeX\ 文档, 你必需运行 \TeX\ 及其相关软件.
你同时也需要一些 \TeX\ 及其相关软件所用到的辅助文件.
在这本书中, 我们仅仅谈论 \TeX\ 和一些通用的软件及辅助文件,
不过我们不会介绍其它的, 因为它们仅仅和你自己的 \TeX\ 环境相关.
为你提供 \TeX\ 的人可以为你提供我们所说的\emph{本地信息}.
\pix^^{local information}
这些本地信息可以告诉你如何去启动 \TeX, 如何去使用相关的程序,
以及如何去访问这些所用到的辅助文件.


你可以使用一个 ^{文本编辑器} 来编写一个可以输入到 \TeX\ 的普通文本文件.
一个 \TeX\ 输入文件和一个文字处理软件的输入文件不同, 它通常并不存在任何不可见的^{控制字符}.
任何 \TeX\ 读到的字符对你来说, 都是你可以看到的.

你的输入文件, 可以仅仅是一个引用其它输入文件的框架.
\TeX\ 用户往往把大的文档, 比如说这本书, 组织成这种形式.
你可以使用 ^|\input| 命令 (\xref\input) 来把一个输入文件篏入到另一个中.
特别地, 你可以使用 |\input|来调入包含\emph{宏定义}的文件,
^^{macros//in auxiliary files}
宏定义是一些辅助的定义, 它可以来增强 \TeX\ 的功能.
如果你的 \TeX\ 系统中包含任何的宏文件,
和 \TeX\ 相关的本地信息会告诉你如何得到这些宏文件,
以及它们的功能.
\TeX\ 的标准安装形式, 也就是在这本书中所描述的, 包括了一个被称为 ^{\plainTeX} (\xref{\plainTeX}) 的宏集.

当 \TeX\ 处理你的文档时, 它会产生一种叫做  ^{\dvifile} 的文件.
其中, ``|dvi|'' 的全称为 ``device independent (设备无关)''.
之所以选用这个缩写, 是因为 \dvifile\ 中所包含的信息和你的打印或显示设备无关.


当你需要打印或使用\emph{阅览器}查看你的文档时, ^^{previewer}
你需要使用相应的\emph{设备驱动}来处理之. ^^{device drivers}
(阅览器是一个程序, 它能把和打印出来的结果差不多的排版内容显示在屏幕上.)
不同的输出设备往往需要不同的设备驱动. 在运行设备驱动以后,
你还需要把设备驱动输出的内容传输到打印机或者其它输出设备上.
^^{printers} ^^{output devices}
和 \TeX\ 相关的本地信息会告诉你如何得到正确的设备驱动和如何使用它.

因为 \TeX\ 内部并没有任何关于一个特定字体的任何信息,
它使用\emph{字体文件}
^^{font files}
来得到你文档中所使用的字体的信息.
字体文件也是你本地 \TeX\ 环境的一个组成部分.
一个字体一般对应两个文件:
一个文件 (\emph{字体信息文件}) ^^{metrics file} 包括了字体中每个字符的大小信息,
另一个文件 (\emph{字体轮廓文件}) ^^{shape file} 则描述了字体中字符的形状.
一个字体的缩放版本使用同样的字体信息文件, 不过并不使用相同的字体轮廓文件.
^^{magnification}
字体信息文件往往用来指称 ^{\tfmfile},
而字体轮廓文件则往往用来指称 ^{\pkfile}, ^{\pxlfile} 和 ^{\gffile} 等不同的种类.
这些名称和 \TeX\ 以及相关软件所使用的文件的文件名所对应.
比如, |cmr10.tfm| 为 |cmr10| 字体的字体信息文件 (10-点的计算机现代字体).

\TeX\ 本身仅使用字体信息文件, 因为它只关心这个字体的字符占用了多大的空间,
而并不关心这个字体的外形是什么样子的.
设备驱动则往往需要使用字体轮廓文件,
因为它需要负责把每个排出的字符变成印出的图形.
某些设备驱动也需要使用字体信息文件.
一些设备驱动可以使用打印机中包含的字体,
所以可以不需要这些字体的字体轮廓文件.
\secondprinting{\vfill\eject}


\subsection 运行 {\TeX}

\bix^^{running \TeX}

你可以通过输入类似 `|run tex|' 或 `|tex|' (查看你的本地信息) 来运行 \TeX\ 来处理一个输入文件 |screed.tex|.
^^{input files}
\TeX\ 会作出如下显示
% 4/23/90 is Shakespeare's 426th birthday, and Karl's 26th.
\csdisplay
This is TeX, Version 3.0 (preloaded format=plain 90.4.23)
**
|
在这里, ``preloaded format'' (已预先载入的格式) 指的是 \TeX\ 提供的 ^{\plainTeX} 宏集的一个己预先产生的形式.
现在你可以输入 `|screed|' 来让 \TeX\ 来处理你的文件.
当处理结束时, 你可以在终端或者打印机印出的记录上 (如果你不使用终端) 得到类似下面的输出:
\csdisplay
(screed.tex [1] [2] [3] )
Output written on screed.dvi (3 pages, 400 bytes).
Transcript written on screed.log.
|
这个输出大部分是不言自明的.
在括号中的数字是 \TeX\ 在把每页内容输入 \dvifile\ 时所显示的当前页码.
当你提供的名件名不含扩展名时, \TeX\ 往往会把你的文件扩展名当作 `|.tex|'.
在某些 \TeX\ 环境中, 你可能可以直接执行类似下面的语句:
\csdisplay
tex screed
|
来直接处理文档.

Instead of providing your \TeX\ input from a file, you can type it directly at
your terminal.  To do so, type `^|\relax|' instead of `|screed|' at the
`|**|' prompt.
\TeX\ will now prompt you with a `|*|' for each line of input and interpret
each line of input as it sees it.
To terminate the input, type a command such as `|\bye|' that tells \TeX\
you're done.
Direct input is sometimes a handy way of experimenting with \TeX.

When your input file contains other embedded input files, the displayed
information indicates when \TeX\ begins and ends processing each
embedded file.
^^{input files//embedded}
\xrdef{infiles}
\TeX\ displays a left parenthesis and the file name
when it starts working on a file and displays the corresponding right
parenthesis when it's done with the file.
If you get any ^{error messages} in the displayed output, you can match
them with a file by looking for the most recent unclosed left parenthesis.

For a more complete explanation of how to run \TeX,
see \knuth{Chapter~6} and your ^{local information}.
\eix^^{running \TeX}


\section Preparing an input file

In this section we explain some of the conventions that you must follow in
preparing input for \TeX\null.  Some of the information given here also
appears in the examples in \chapterref{examples} of this book.
^^{input, preparing}

\subsection Commands and control sequences

\bix^^{commands}
\bix^^{control sequences}
Input to \TeX\ consists of a sequence of commands that tell \TeX\ how to
typeset your document.  Most characters act as commands of a particularly
simple kind: ``typeset me''.  The letter `|a|', for instance, is a
command to typeset an `a'.  But there's another kind of command---a
\emph{control sequence}---that gives \TeX\ a more elaborate
instruction.  A control sequence ordinarily starts with a backslash
(|\|), though you can change that convention if you need to.
\xrdef{@backslash}
For instance, the input:

\csdisplay
She plunged a dagger (\dag) into the villain's heart.
|
contains the control sequence |\dag|; it produces the typeset output:
\display{%
She plunged a dagger (\dag) into the villain's heart.
}
\noindent Everything in this example except for the |\dag| and the spaces
acts like a ``typeset me'' command.  We'll explain more about spaces
on \xrefpg{spaces}.

There are two kinds of control sequences: \emph{control words}
^^{control words}
and \emph{control symbols}:
^^{control symbols}
\ulist\compact
\li A control word consists of a
backslash followed by one or more letters, e.g., `|\dag|'.
The first character that isn't a letter marks the end
of the control word.
\li A control symbol consists of a backslash followed by a single character
that isn't a letter, e.g., `|\$|'.
The character can be a space or even the end of a line (which is a perfectly
legitimate character).
\endulist
\noindent
A control word (but not a control symbol)
absorbs any spaces or ends of line that follow it.
^^{control sequences//absorbing spaces}
If you don't want to lose a space after a control word,
follow the control sequence with a ^{control space}
(|\!visiblespace|) or with `|{}|'.  Thus either:
\csdisplay
The wonders of \TeX\!visiblespace!.shall never cease!!
|
or:\hfil\
\csdisplay
The wonders of \TeX{} shall never cease!!
|
produces:
\display{%
The wonders of \TeX{} shall never cease!
}
\noindent rather than:
\display{%
The wonders of \TeX shall never cease!
}
\noindent
which is what you'd get if you left out the `|\|\visiblespace'
or the `|{}|'.

Don't run a control word together with the text that follows it---\TeX\
won't know where the control word ends.  For instance, the |\c| control
sequence places a cedilla accent on the character that follows it.  The
French word {\it gar\c con\/} must be typed as
`|gar\c!visiblespace!.con|', not `|gar\ccon|'; if you write the latter,
\TeX\ will complain about an undefined control sequence |\ccon|.

A control symbol, on the other hand, doesn't absorb anything that
follows it.  Thus you must type `\$13.56' as `|\$13.56|', not
`|\$!vs13.56|'; the latter form would produce `\hbox{\$ 13.56}'.
However, those accenting commands that are named by control symbols are
defined in such a way that they produce the effect of absorbing a
following space.  Thus, for example, you can type the French word {\it
d\'eshabiller\/} either as `|d\'eshabiller|' or as
`|d\'!visiblespace!.eshabiller|'.

Every control sequence is also a command,
but not the other way around.
^^{commands//versus control sequences}
^^{control sequences//versus commands}
For instance, the letter `|N|'
is a command, but it isn't a control sequence.
In this book we ordinarily use ``command'' rather than
``control sequence'' when either term would do.
We use ``control sequence'' when we want to emphasize aspects of \TeX\
syntax that don't apply to commands in general.

\eix^^{control sequences}
\eix^^{commands}


\subsection Arguments

\xrdef{arg1}
Some commands need to be followed by one or more
\emph{arguments} ^^{arguments}
that help to determine what the command does.
For instance, the |\vskip| command, which
tells \TeX\ to skip down (or up) the page,
expects an argument specifying how much space to skip.  To skip
down two inches, you would type `|\vskip 2in|', where |2in|
is the argument of |\vskip|.

Different commands expect different kinds of arguments.  Many commands
expect dimensions, such as the |2in| in the example above.
Some commands, particularly those defined by macros,
expect arguments that are either a single character or some
text enclosed in braces.
Yet others require that their arguments be enclosed in braces, i.e.,
they don't accept single-character arguments.
The description of each command in this book tells you what kinds of arguments,
if any, the command expects.
In some cases, required braces define a group (see \xref{bracegroup}).

\secondprinting{\vfill\eject}


\subsection Parameters

\xrdef{introparms}
Some commands are parameters (\xref{parameter}).
^^{parameters//as commands}
You can use a parameter in either of two ways:
\olist
\li You can use the value of a parameter
as an argument to another command.  For example, the command
\hbox{|\vskip\parskip|}
causes a vertical skip by the value of the |\parskip| (paragraph skip)
glue parameter.
\li You can change the value of the parameter by assigning
something to it.  For example, the assignment \hbox{|\hbadness=200|}
causes the value of the |\hbadness| number parameter to be $200$.
\endolist
\noindent
We also use the term ``parameter'' to refer to entities such as |\pageno|
that are actually registers but behave just like parameters.
^^{registers//parameters as}

Some commands are names of tables.  These commands are used like
parameters, except that they require an additional argument that
specifies a particular entry in the table. For example, |\catcode| names
a table of category codes (\xref{category code}). Thus
the command
\hbox{|\catcode`~=13|} sets the category code of the `|~|'
character to $13$.


\subsection Spaces

\xrdef{spaces}
\bix^^{spaces}
You can freely use extra spaces in your input.  Under nearly all circumstances
\TeX\ treats several spaces in a row as being equivalent to a
single space.  For instance, it doesn't matter whether you put one space
or two spaces after a ^{period} in your input.  Whichever you do, \TeX\
performs its end-of-sentence maneuvers and leaves the appropriate
(in most cases) amount of space after the period.
\TeX\ also treats the end of an input line as equivalent to a space.
Thus you can end your input lines wherever it's convenient---%
\TeX\ makes input
lines into
paragraphs in the same way no matter where the line breaks are in your
input.

A blank line in your input marks the end of a paragraph.
^^{paragraphs//ending}
Several blank lines are equivalent to a single one.

\TeX\ ignores input spaces within math formulas (see below).  Thus you can
include or omit spaces anywhere within a math formula---\TeX\ doesn't care.
Even within a math formula, however,
you must not run a control word together with a following letter.

If you are defining your own macros, you need to be particularly careful about
where you put ends of line in their definitions.
It's all too easy to define a macro that produces an
^{unwanted space} in addition to whatever else it's supposed to produce.
We discuss this problem elsewhere since it's somewhat
technical; see \xrefpg{unwantedspace}.

A space or its equivalent between two words in your input doesn't simply turn
into a space character in your output.
A few of these input spaces turn into ends of lines
in the output,
since input lines generally don't correspond to output lines.
The others turn into spaces of variable width called ``glue'' (\xref{glue}),
which has a natural size (the size it ``wants to be'')
but can stretch or shrink.
When \TeX\ is typesetting a paragraph
that is supposed to have an even right margin (the usual
case), it adjusts the widths of the glue in each line
to get the lines to end at the margin.
(The last line of a paragraph is an exception, since it isn't ordinarily
required to end at the right margin.)

You can prevent an input space from turning into an end of line by using a
^{tie} (^|~|).
For example, you wouldn't want \TeX\ to put a line break between the
`Fig.' and `8' of `Fig.~8'.
By typing `|Fig.~8|' you can prevent such a line break.
\eix^^{spaces}
\needspace{2in}

\subsection Comments

\xrdef{comments}
\pix\bix^^{comments}
You can include comments in your \TeX\ input.
When \TeX\ sees a comment it just passes over it, so
what's in a comment doesn't affect your typeset document in any way.
Comments are useful for
providing extra information about what's in your input file.
For example:
\csdisplay
% ========= Start of Section `Hedgehog' =========
|

{\indexchar % }%
A comment starts with a percent sign (|%|) and extends to the end of the
input line.
\TeX\ ignores not just the comment but the end of the line as well, so
comments have another very
important use: connecting two lines so that the end of line
^^{line breaks//deleting}
between them is invisible to \TeX\ and doesn't generate
an output space or an end of line.
For instance, if you type:
\csdisplay
A fool with a spread%
sheet is still a fool.
|
you'll get:
\display{
A fool with a spread%
sheet is still a fool.
}
\eix^^{comments}


\subsection Punctuation

\null
\xrdef{periodspacing}
\TeX\ normally adds some extra space after what it thinks is a
^{punctuation} mark at the end of a sentence,
namely, `^|.|', `^|?|', or `|!!|' \indexchar !
^^{period} ^^{question mark} ^^{exclamation point}
followed by an input space.
\TeX\ doesn't add
the extra space if the punctuation mark follows
a capital letter, though, because it assumes the capital
letter to be an initial in someone's name.
You can force the extra space where it wouldn't otherwise occur by
typing something like:
\csdisplay
A computer from IBM\null?
|
The |\null| doesn't produce any output, but it does prevent \TeX\
from associating the capital `M' with the question mark.
On the other hand, you can cancel the
extra space where it doesn't belong by typing a control space
after the punctuation mark, e.g.:
\csdisplay
Proc.\!visiblespace!.Royal Acad.\!visiblespace!.of Twits
|
so that you'll get:
\display{Proc.\ Royal Acad.\ of Twits}
\noindent rather than:
\display{Proc. Royal Acad. of Twits}

Some people prefer not to leave more space after punctuation at the
end of a sentence.  You can get this effect with the
^|\frenchspacing| command (\xref\frenchspacing).
|\frenchspacing| is often recommended for ^{bibliographies}.

For single ^{quotation marks}, you should use the left and right
single quotes
(|`| and |'|) on your keyboard.  For left and right
double quotation marks, use two left single
quotes or two right single quotes (|``| or |''|) rather
than the double quote (|"|) on your keyboard.
The keyboard double quote
will in fact give you a right double quotation mark in
many fonts, but the two right single quotes
are the preferred \TeX\ style.
For example:

\vbox{%
\csdisplay
There is no `q' in this sentence.
``Talk, child,'' said the Unicorn.
She said, ``\thinspace`Enough!!', he said.''
|
}%
These three lines yield:
\display{\par\restoreplainTeX
There is no `q' in this sentence.
\par ``Talk, child,'' said the Unicorn.
\par She said, ``\thinspace`Enough!', he said.''
}
\noindent
The |\thinspace| in the third input line prevents
the single quotation mark from coming
too close to the double quotation marks.
Without it, you'd just see three
nearly equally spaced quotation marks in a row.

\TeX\ has three kinds of ^{dashes}:
\ulist\compact
\li Short ones (hyphens) like this ( - ). You get them by typing~`^|-|'.
\li Medium ones (en-dashes) like this ( -- ). You get them by typing~`^|--|'.
\li Long ones (em-dashes) like this ( --- ). You get them by typing~`^|---|'.
\endulist
\noindent
Typically you'd use hyphens to indicate compound words like
``will-o'-the-wisp'',
en-dashes to indicate
page ranges such as ``pages~81--87'', and em-dashes to indicate
a break in continuity---like this.


\subsection Special characters

Certain characters have special meaning to \TeX, so you shouldn't use them
in ordinary text.  They are:

\csdisplay
    $  #  &  %  _  ^  ~  {  }  \
|
^^|$//in ordinary text|
^^|#//in ordinary text|
^^|&//in ordinary text|
^^|_//in ordinary text|
^^|^//in ordinary text|
^^|~//in ordinary text|
^^|%//in ordinary text|
^^|{//in ordinary text|
^^|}//in ordinary text|
{\recat!ttidxref[\//in ordinary text]]
\noindent
In order to produce them in your typeset document,
you need to use circumlocutions.  For the first five,
you should instead type:
^^|\$|
^^|\#|
^^|\&|
^^|\%|
^^|\_|
\csdisplay
    \$  \#  \&  \%  \_
|

\noindent
For the others, you need something more elaborate:

\csdisplay
   \^{!visiblespace}   \~{!visiblespace}   $\{$   $\}$   $\backslash$
|


\subsection Groups

\bix^^{groups}
A \emph{group}
consists of material enclosed in matching left and right braces (|{| and
|}|).
^^|{//starting a group|
^^|}//ending a group|
By placing a command within a group, you can limit its effects to
the material within the group.
For instance, the |\bf| command tells \TeX\ to set
something in {\bf boldface} type.  If you were to put |\bf| into your input
and do nothing else to counteract it, everything in your document following the
|\bf| would be set in boldface.
By enclosing |\bf| in a group,
you limit its effect to the group.  For example, if you type:
\csdisplay
We have {\bf a few boldface words} in this sentence.
|
\noindent you'll get:
\display{We have {\bf a few boldface words} in this sentence.}

\noindent You can also use a group to limit the effect of
an assignment to one of \TeX's parameters.
These parameters contain values that affect how \TeX\ typesets your document.
For example, the value of the |\parindent|
parameter specifies the indentation at the beginning of a paragraph.
The assignment |\parindent = 15pt|
sets the indentation to $15$ printer's points.
By placing this assignment at the beginning
of a group containing a few paragraphs, you can change
the indentation of just those paragraphs.  If you don't enclose
the assignment in a group,
the changed indentation will apply to the rest of the document (or up to the
next assignment to |\parindent|, if there's a later one).

\xrdef{bracegroup}
Not all pairs of braces indicate a group.
In particular, the braces associated with an argument for which the
braces are \emph{not} required don't indicate a group---they just
serve to delimit the argument.
Of those commands that do require braces for their arguments,
some treat the braces as defining a group
and the others interpret the argument in some special way that depends on
the command.\footnote
{More precisely, for primitive commands either
the braces define a group or they enclose tokens that aren't processed in
\TeX's stomach.
For |\halign| and |\valign| the group has a trivial
effect because everything within the braces either doesn't reach the stomach
(because it's in the template) or is enclosed in a further inner group.
^^|\halign//grouping for|
^^|\valign//grouping for|
}
\eix^^{groups}


\subsection Math formulas

\bix^^{math}
\xrdef{mathform}
A math formula can appear in text (\emph{text math})
^^{text math}
or set off on a line by itself
with extra vertical space around it (\emph{display math}).
^^{display math}
You enclose a text formula in single dollar signs (|$|)
and a displayed formula in double dollar signs (|$$|).
\ttidxref{$}\ttidxref{$$}
For example:

\csdisplay
If $a<b$, then the relation $$e^a < e^b$$ holds.
|
\noindent This input produces:
\display{\centereddisplays
If $a<b$, then the relation $$e^a < e^b$$ holds.}
\smallskip
\noindent \chapterref{math} describes the commands that are useful
in math formulas.
\eix^^{math}


\section How \TeX\ works

In order to use \TeX\ effectively, it helps to
have some idea of how \TeX\ goes about
its activity of transmuting input into output.
You can imagine \TeX\ as a kind of organism with ``eyes'',
``mouth'', ``gullet'',
``stomach'', and ``intestines''.
Each part of the organism transforms its input in some way and passes
the transformed input to the next stage.

The ^{eyes} transform an input file into a sequence of characters.
The ^{mouth} transforms the sequence of characters into a sequence of
\emph{tokens},
^^{tokens}
where each token is either a single character or a control sequence.
^^{control sequences//as tokens}
The gullet expands the tokens into a sequence of
\emph{primitive commands}, which are also tokens.
^^{expanding tokens}
The ^{stomach} carries out the operations specified by the primitive commands,
producing a sequence of pages.
Finally, the ^{intestines} transform each page into the form required
for the \dvifile\ and send it there.
^^{\dvifile//created by \TeX's intestines}
These actions are described in more detail
in \chapterref{concepts} under \conceptcit{\anatomy}.
^^{\anatomy}

The real typesetting goes on in the stomach.
The commands instruct \TeX\ to typeset such-and-such a character in
such-and-such a font, to insert an interword space, to end a paragraph, and
so on.
Starting with individual typeset characters and other simple typographic
elements, \TeX\ builds up a page ^^{pages} as a nest of
^{boxes} within boxes within boxes \seeconcept{box}.
Each typeset character occupies a box, and so does an entire page.
A box can contain not just smaller boxes but also \emph{glue} ^^{glue}
(\xref{glue}) and a few other things.
The glue produces
space between the smaller boxes.
An important property of glue is that it can stretch and shrink;
thus \TeX\ can make a box
larger or smaller by stretching or shrinking
the glue within~it.

Roughly speaking, a line is a box containing a sequence of character boxes,
and a page is a box containing a sequence of line boxes.
There's glue between the words of a line and between the lines of a page.
\TeX\ stretches or shrinks
the glue on each line so as to make the right margin
of the page come out even and the glue on each page
so as to make the bottom margins of different pages be equal.
Other kinds of typographical elements can also appear in a line or in a page,
but we won't go into them here.

As part of the process of assembling pages, \TeX\ needs to break paragraphs
into lines and lines into pages.  The stomach first sees a paragraph as one
long line, in effect.  It inserts \emph{line breaks}
^^{line breaking}
in order to transform
the paragraph into a sequence of lines of the right length, performing a
rather elaborate analysis in order to choose the set of breaks
that makes the paragraph look best
\seeconcept{line break}.
The stomach carries out a similar
but simpler process in order to transform a sequence of lines into a page.
Essentially the stomach accumulates lines until no more lines can fit on the
page.  It then chooses a single place to break the page, putting the lines
before the break on the current page
and saving the lines after the break for the
next page \seeconcept{page break}. ^^{page breaks//inserted by \TeX's stomach}

When \TeX\ is assembling an entity from a list of items (boxes, glue, etc.),
it is in one of six
\emph{modes} ^^{modes} (\xref{mode}).
The kind of entity it is assembling defines the mode that it is in.
There are two ordinary modes: ordinary horizontal mode for assembling
paragraphs (before they are broken into lines)
and ordinary vertical mode for assembling pages.
There are two restricted modes:
restricted horizontal mode for appending items horizontally to form
a horizontal box
and internal vertical mode for appending items vertically to form
a vertical box (other than a page).
Finally, there are two math modes: text math mode for assembling math formulas
within a paragraph and display math mode for assembling math formulas that are
displayed on lines by themselves (see ``Math formulas'', \xref{mathform}).


\section 新 \TeX\ 和老 {\TeX}

\xrdef{newtex}
在 1989 年, Knuth 对 \TeX\ 做了一次大幅的修改, 来使它可以处理非英语排版所需用的字符集.\space ^^{foreign languages}
这次修改还包括了一些额外的不干扰其它东西的小功能,
这本书介绍 ``^{\newTeX}''.
This book describes ``^{\newTeX}''.
如果你仍使用旧版本的 \TeX\ ($2.991$ 版或以前的版本)
你可能会想知道哪些 {\newTeX} 的功能你是没法使用的.
以下这些功能无法在旧版中使用:
\ulist\compact
\li ^|\badness| (\xref\badness)
\li ^|\emergencystretch| (\xref\emergencystretch)
\li ^|\errorcontextlines| (\xref\errorcontextlines)
\li ^|\holdinginserts| (\xref\holdinginserts)
\li ^|\language|, ^|\setlanguage|, 和 |\new!-lan!-guage|
(\pp\xrefn\language, \xrefn{\@newlanguage}) ^^|\newlanguage|
\li ^|\lefthyphenmin| 和 ^|\righthyphenmin| (\xref\lefthyphenmin)
\li ^|\noboundary| (\xref\noboundary)
\li ^|\topglue| (\xref\topglue)
\li 十六进制的数字表达方式 |^^|$xy$  (\xref{hexchars})
\endulist
\noindent
我们建议你尽可能地使用新版本的 \TeX.

\section 资源

\xrdef{resources}
有很多资源可以帮助你使用 \TeX.
\texbook\ 是最可靠的 \TeX\ 信息来源.

\smallskip
{\narrower\noindent
^{Knuth, Donald E.}, \texbook.  Reading, Mass.: Addison-Wesley, 1984.\par}
\smallskip
\noindent
请一定使用第十七次 (1990 年 1 月) 及以后的印刷版本.
早先的印刷版本不包括新 \TeX\ 的许多功能.

^{\LaTeX} 是一套为了简化使用 \TeX\ 而设计的命令集
在这本书中有描述\footnote{译者注: 这本书中介绍的 \LaTeX\ 己经过时}:
\smallskip
{\narrower\noindent\frenchspacing\spaceskip = 3.33pt plus 2pt minus 1.2pt
^{Lamport, Leslie}, {\sl The \LaTeX\ Document Preparation System}.
Reading, Mass.: Addison-Wesley, 1986.\par}
\smallskip
\noindent
^{\AMSTeX} 是一套美国数学学会定义的提交电子数学手稿的命令标准,
在这里有描述:
\smallskip
{\narrower\noindent
^{Spivak, Michael~D.}, {\sl The Joy of \TeX}. Providence, R.I.:
American Mathematical Society, 1986.
\par}
\smallskip
\noindent
你可以加入 ^{\TUG} (TUG), 这个组织出版一本名为 {\it ^{TUGBoad}} 的通讯.
TUG 是一个很好的信息来源渠道, 它同时也提供包括 \AMSTeX\ 在内的很多 \TeX\ 宏包资源,
它的地址是:
\smallskip
{\obeylines
^{\TUG}
c/o American Mathematical Society
P.O. Box 9506
Providence,  RI  02940
U.S.A.
}
\smallskip
\noindent
最后, 你可以获取在 \chapterref{eplain} 介绍的用来排版本书的宏包 ^|eplain.tex|.
它可以通过使用 \ftp\ 匿名登录获取:
{\obeylines\display{\tt
labrea.stanford.edu [36.8.0.47]
ics.uci.edu [128.195.1.1]
june.cs.washington.edu [128.95.1.4]}}

电子版本还包括了斯坦弗大学的 Oren Patashnik ^^{Patashnik, Oren} 为 ^{\BibTeX} 计算机程序处理输入并且排印出其输出的宏.
如果你在这个宏包中发现了错误, 或者想改进它, 你可以给 Karl 发电子邮件, Karl 的电子邮箱地址是 {\tt karl@cs.umb.edu}.

这个宏包可以通过向以下地址邮寄 10.00 美元购买 $5\frac1/4$\inches\
或 $3\frac1/2$\inches\  软盘得到.
\smallskip
{\obeylines
Paul Abrahams
214 River Road
Deerfield,  MA  01342
\vskip\tinyskipamount
Email: {\tt Abrahams\%Wayne-MTS@um.cc.umich.edu}
}
\smallskip
\noindent
这个地址在 1990 年 6 月是正确的; 请注意在这之后它可能发生改变, 尤其是电子邮件地址.

\endchapter\byebye
