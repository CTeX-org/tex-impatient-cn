% -*- coding: utf-8 -*-
% This is part of the book TeX for the Impatient.
% Copyright (C) 2003 Paul W. Abrahams, Kathryn A. Hargreaves, Karl Berry.
% See file fdl.tex for copying conditions.

\input macros
\chapter {组页命令}

\chapterdef{pages}

这个章节介绍了与页, 页的组成部分以及输出程序相关的命令.
本章写作惯例的说明可以在\headcit{命令描述}{cmddesc}中找到.

\begindescriptions

%==========================================================================
\section {行间距和段间距}

\begindesc
\cts baselineskip {\param{glue}}
\cts lineskiplimit {\param{dimen}}
\cts lineskip {\param{glue}}
\explain
%^^{line spacing}
%\bix^^{interline glue}
%\bix^^|\baselineskip|
%\bix^^|\lineskip|
%\bix^^|\lineskiplimit|
%These three parameters jointly determine how much space \TeX\ leaves between
%consecutive \minref{box}es of an ordinary \minref{vertical list},
%e.g., the lines of a paragraph.
%This space is called ``\minref{interline glue}''.
%It is also inserted between the component boxes of a vbox constructed in
%internal vertical mode.
%^^{vboxes//interline glue for}
%
%In the usual case, when the boxes aren't abnormally high or deep, \TeX\
%makes the distance from the baseline of one box to the baseline of the
%next one equal to |\baselineskip|.  It does this by inserting interline
%glue equal to |\baselineskip| minus the depth of the upper box (as given
%by ^|\prevdepth|) and the height of the lower box.  But if this
%interline glue would be less than |\lineskiplimit|, indicating that the
%two boxes are too close together, \TeX\ inserts the |\lineskip| glue
%instead.\footnote
%{\TeX\ actually accounts for the beginning of a
%vertical list by setting |\prevdepth| to $-1000$\pt\ and testing
%|\prevdepth| before \emph{every} box.  If |\prevdepth|$\>\le-1000$\pt\
%it does not insert any interline glue.} See \knuth{pages~79--80} for a
%precise description.
%
%Note that |\baselineskip| and |\lineskip| measure \emph{different
%things}: the distance between baselines on the one hand and the distance
%between the bottom of one box and the top of the next box on the other
%hand.  See \knuth{page~78} for further details.  The first example below
%shows the effects of |\lineskiplimit|.
%
%You can obtain the effect of ^{double spacing} by doubling the value
%of |\baselineskip| as illustrated in the second example below.
%A change to |\baselineskip| at any point before the end of a paragraph affects
%the entire paragraph.
^^{line spacing}
\bix^^{interline glue}
\bix^^|\baselineskip|
\bix^^|\lineskip|
\bix^^|\lineskiplimit|
在一个普通\minref{垂直列}(例如段落各行)中,
这三个参数共同确定了 \TeX\ 在相邻两个\minref{盒子}间插入的间隔。
这个间隔称为“\minref{行间\zglue}”。
对于内部垂直模式中构造的 vbox 垂直列,其中的组成盒子间同样会有此间隔。
^^{vboxes//interline glue for}

在通常情况下,盒子的高度和深度不会太大,
此时 \TeX\ 让相邻盒子的两基线距离等于 |\baselineskip|。
为此可以让行间\zglue 等于 |\baselineskip| 减去上盒子深度(由 ^|\prevdepth| 给出)%
和下盒子高度。
但如果两个盒子靠得太近,即行间\zglue 小于 |\lineskiplimit|,
\TeX\ 将改为插入 |\lineskip| 大小的间距。\negthinspace\footnote
{\TeX\ 实际上在垂直列开头设定 |\prevdepth| 为 $-1000$\pt ,
从而可以测定\emph{每个}盒子的 |\prevdepth|。
若 |\prevdepth|$\>\le-1000$\pt ,
它就不会插入任何行间\zglue 。}详情可见 \knuth{第~79--80~页}。

注意 |\baselineskip| 和 |\lineskip| 测量\emph{不同的东西}:
前者为两盒子基线之间的距离,
而后者为上盒子底部和下盒子顶部的距离。
详情可看 \knuth{第~78~页}。
下面第一个例子展示了|\lineskiplimit| 的作用。

如下面第二个例子所示,将 |\baselineskip| 加倍,
就能得到^{双倍行距}的效果。
在段落结束前任何位置改变 |\baselineskip|,都会影响整个段落。
\example
\baselineskip = 11pt \lineskiplimit = 1pt
\lineskip = 2pt plus .5pt
Sometimes you'll need to typeset a paragraph that has
tall material, such as a mathematical formula,  embedded
within it.  An example of such a formula is $n \choose k$.
Note the extra space above and below this line as
compared with the other lines.
(If the formula didn't project below the line,
we'd only get extra space above the line.)
|
\produces
\baselineskip = 11pt \lineskiplimit = 1pt
\lineskip = 2pt plus .5pt
Sometimes you'll need to typeset a paragraph that has
tall material, such as a mathematical formula,  embedded
within it.  An example of such a formula is $n \choose k$.
Note the extra space above and below this line as
compared with the other lines.
(If the formula didn't project below the line,
we'd only get extra space above the line.)
\endexample

\example
\baselineskip = 2\baselineskip % Start double spacing.
|
\endexample

\eix^^{interline glue}
\eix^^|\baselineskip|
\eix^^|\lineskip|
\eix^^|\lineskiplimit|
\enddesc

\begindesc
\cts prevdepth {\param{dimen}}
\explain
%When \TeX\ adds a box to a vertical list, it sets |\prevdepth| to the
%depth of that box.  \TeX\ sets |\prevdepth| to $-1000$\pt\ at the start
%of a vertical list, indicating that the usual interline glue should be
%suppressed.
当 \TeX\ 将盒子添加到垂直列时,它设定 |\prevdepth| 为该盒子的深度。
在垂直列的开始, \TeX\ 设定 |\prevdepth| 为 $-1000$\pt\ ,这表示不添加行间\zglue 。
\enddesc

\begindesc
\cts normalbaselineskip {\param{glue}}
\cts normallineskiplimit {\param{dimen}}
\cts normallineskip {\param{glue}}
\cts normalbaselines {}
\explain
%The three parameters contain values for |\base!-line!-skip|,
%|\line!-skip|, and |\line!-skip!-limit| respectively.
%^^|\baselineskip| ^^|\lineskip| ^^|\lineskiplimit|
%The |\normalbaselines| command sets |\base!-line!-skip|,
%|\line!-skip|, and |\line!-skip!-limit| to the values contained in the
%three parameters.
前三个参数分别包含 |\base!-line!-skip|,|\line!-skip| 和 |\line!-skip!-limit| 的值。
^^|\baselineskip| ^^|\lineskip| ^^|\lineskiplimit|
而 |\normalbaselines| 命令将 |\base!-line!-skip|,|\line!-skip| 和 |\line!-skip!-limit|
分别设定为这三个参数的值。
\enddesc

\begindesc
\cts offinterlineskip {}
\explain
%This command tells \TeX\ to stop inserting interline glue from
%now on.  Unless you want it to be in effect for the rest of the document
%(which you probably don't), you should enclose it in a group
%together with the text you want it to affect.
%Its main purpose is to let you do interline spacing yourself,
%e.g., using ^{struts},
%without interference from \TeX's normal interline glue.
%|\offinterlineskip| is often useful when
%you're constructing a horizontal \minref{alignment}.
%^^{alignments//using \b\tt\\offinterlineskip\e\ in}
此命令让 \TeX\ 从此之后停止插入行间\zglue 。除非你希望对文档的剩余部分都生效(你多半不希望),
你应该将此命令和你希望生效的文本放在一个\zgroup 中。
它的主要用处在于可以让你自己处理行间空隙,比如用支架(^{struts}),
而不影响 \TeX\ 的正常行间\zglue 。在构造水平对齐时经常会用到|\offinterlineskip|。
^^{alignments//using \b\tt\\offinterlineskip\e\ in}
\example
\def\entry#1:#2 {\strut\quad#1\quad&\quad#2\quad\cr}
\offinterlineskip \tabskip = 0pt \halign{%
\vrule\quad\hfil#\hfil\quad\vrule&
   \quad\hfil#\hfil\quad\vrule\cr
\noalign{\hrule}
\vphantom{\vrule height 2pt}&\cr \noalign{\hrule}
\entry \it Opera:\it Composer
\vphantom{\vrule height 2pt}&\cr \noalign{\hrule}
\vphantom{\vrule height 2pt}&\cr
\entry Fidelio:Beethoven
\entry Peter Grimes:Britten
\entry Don Giovanni:Mozart
\vphantom{\vrule height 2pt}&\cr \noalign{\hrule}}
|
\produces
\medskip
\def\entry#1:#2 {\strut\quad#1\quad&\quad#2\quad\cr}
\offinterlineskip \tabskip = 0pt \halign{%
\vrule\quad\hfil#\hfil\quad\vrule&
   \quad\hfil#\hfil\quad\vrule\cr
\noalign{\hrule}
\vphantom{\vrule height 2pt}&\cr \noalign{\hrule}
\entry \it Opera:\it Composer
\vphantom{\vrule height 2pt}&\cr \noalign{\hrule}
\vphantom{\vrule height 2pt}&\cr
\entry Fidelio:Beethoven
\entry Peter Grimes:Britten
\entry Don Giovanni:Mozart
\vphantom{\vrule height 2pt}&\cr \noalign{\hrule}}
\endexample
\enddesc

\begindesc
\cts nointerlineskip {}
\explain
%This command tells \TeX\ not to insert interline glue in front of the
%next line.
%It has no effect on subsequent lines.
此命令让 \TeX\ 停止在下一行插入行间\zglue 。它不影响后续其他行。
\enddesc

\begindesc
\cts openup {\<dimen>}
\explain
%This command increases ^|\baselineskip| by \<dimen>.
%An |\openup| command before the end of a paragraph affects
%the entire paragraph, so you shouldn't use |\openup| to
%change |\baseline!-skip| within a paragraph.  |\openup| is
%most useful for typesetting tables and math displays---a
%little extra space between rows often makes them more readable.
%^^{alignments//space between rows of}
此命令让 ^|\baselineskip| 增加 \<dimen>。
在段尾之前的 |\openup| 命令将影响整个段落,
因此想在段落中间改变 |\baseline!-skip| 时不要用 |\openup| 命令。
在排版表格和展示公式时 |\openup| 非常有用——多点间隔将会增加它们的可读性。
^^{alignments//space between rows of}
\example
Alice picked up the White King very gently, and lifted him
across more slowly than she had lifted the Queen; but before
she put him on the table, she thought she might well dust
him a little, he was so covered with ashes.
\openup .5\baselineskip % 1.5 linespacing.
|
\produces
Alice picked up the White King very gently, and lifted him
across more slowly than she had lifted the Queen; but before
she put him on the table, she thought she might well dust
him a little, he was so covered with ashes.
\openup .5\baselineskip %1.5 linespacing
\endexample\enddesc

%==========================================================================
\section{分页} %\section {Page breaks}

%==========================================================================
%\subsection {Encouraging or discouraging page breaks}
\subsection{鼓励或阻碍分页}

\begindesc
\bix^^{page breaks}
\bix^^{page breaks//encouraging or discouraging}
\ctspecial break {} \xrdef{vbreak}
\explain
%
\margin{Four commands identical to ones for line breaks (\xref{hbreak})
have been added to correct an omission. The descriptions are exactly parallel.}
%
%This command forces a page break.
%Unless you do something to fill out the page, you're likely to
%get an underfull vbox.
%|\break| can also be used in horizontal mode.
此命令强制在此处分页。除非用某种方式填满该页,你将得到一个松散的 vbox。
|\break| 也能用在水平模式中。
\enddesc

\begindesc
\ctspecial nobreak {} \xrdef{vnobreak}
\explain
%This command prevents a page break where it
%otherwise might occur.
%|\nobreak| can also be used in horizontal mode.
此命令阻止在此处分页。
|\nobreak| 也能用在水平模式中。
\enddesc

\begindesc
\ctspecial allowbreak {} \xrdef{vallowbreak}
\explain
%This command tells \TeX\ to
%allow a page break where one could not ordinarily occur.
%|\allowbreak| can also be used in horizontal mode.
在通常不分页的地方允许 \TeX\ 分页。
|\allowbreak| 也能用在水平模式中。
\enddesc

\begindesc
\ctspecial penalty {\<number>} \xrdef{vpenalty}
\explain
%This command produces a \minref{penalty} item.
%The penalty item makes \TeX\ more or less willing to break a page
%at the point where that item occurs.
%A negative penalty, i.e., a bonus, encourages a page break;
%a positive penalty discourages a page break.
%A penalty of $10000$ or more prevents a break altogether,
%while a penalty of $-10000$ or less forces a break.
%|\penalty| can also be used in horizontal mode.
此命令生成一个惩罚(\minref{penalty})项。
惩罚项使得 \TeX\ 或多或少愿意在此处分页。
负惩罚值,实际上是奖励值,鼓励分页;正惩罚值阻碍分页。
大于或等于 $10000$ 的惩罚值彻底阻止分页,
而小于或等于 $-10000$ 的惩罚值强制分页。
此命令也能用于水平模式。
\example
\def\break{\penalty-10000 } % as in plain TeX
\def\nobreak{\penalty10000 } % as in plain TeX
\def\allowbreak{\penalty0 } % as in plain TeX
|
\endexample
\enddesc

\begindesc
\cts goodbreak {}
\explain
%This command ends a paragraph and also indicates to \TeX\ that
%this is a good place to break the page.
此命令结束当前段落,并告诉 \TeX\ 这里是一个合适的分页位置。
\enddesc

\begindesc
\cts smallbreak {}
\cts medbreak {}
\cts bigbreak {}
\explain
%These commands indicate increasingly desirable places for \TeX\ to
%break a page.  They also cause \TeX\ to
%insert a ^|\smallskip|, ^|\medskip|, or
%^|\bigskip| (\xref \smallskip) if the page break doesn't actually
%happen.  \TeX\ suppresses this skip if it occurs just after an
%equal or larger skip.
这些命令指明了越来越合适的分页位置。如果实际上没在此处分页,\TeX\ 将分别
插入大小为 ^|\smallskip|,^|\medskip| 或 ^|\bigskip| 的间距(\xref \smallskip )。
但如果此位置之前已经有大于或等于此值的间距,\TeX\ 就不会再插入间距。
\enddesc

\begindesc
\easy\cts eject {}
\cts supereject {}
\explain
%These commands force a page break at the current position
%and end the current paragraph.
%If you don't precede them with |\vfil| (\xref \vfil),
%^^|\vfil//needed with {\tt\\eject}|
%\TeX\ will try to stretch out the page contents
%(and will probably complain about an underfull vbox).
%The |\supereject| command, in addition,
%instructs the \plainTeX\ output routine to
%force out any leftover insertions,
%^^{insertions//forced out by \b\tt\\supereject\e}
%such as long footnotes,
%so that they are produced before any more input
%is processed. Thus |\supereject| is a good command to use at the end of each
%chapter or other major unit of your document.
这两个命令将结束当前段落并强制在当前位置分页。如果你不在前面加上
|\vfil| (\xref \vfil),^^|\vfil//needed with {\tt\\eject}|
\TeX\ 将试着将伸展页面内容(很可能会警告有松散的垂直盒子)。
|\supereject| 命令还让 \plainTeX\ 输出例行程序处理剩下的插入项,
^^{insertions//forced out by \b\tt\\supereject\e}例如长脚注,
在处理其他输入之前生成它们。因而,
|\supereject| 更适合在文档各章或主要单元结束时使用。
\enddesc

\begindesc
\cts filbreak {}
\explain
%This command provides a kind of conditional page break.
%It tells \TeX\ to
%break the page---but not if the text up to a later |\filbreak| also
%fits on the same page.
%By enclosing a paragraph in a pair of |\filbreak|s,
%you can ensure that \TeX\ will keep a paragraph on a single page if it can.
%You should not use |\filbreak| within a paragraph, since it
%forces \TeX\ into vertical mode and thus ends the paragraph.
%See \xrefpg{filbreak} for more advice on this subject.
此命令规定了有条件的分页。它告诉 \TeX\ ,如果后面的 |\filbreak| 不能放在同一页,就在此处分页。
用一对 |\filbreak| 将一个段落围起来,就可以确保 \TeX\ 尽量将该段落放在同一页。
不要在段落中间使用 |\filbreak|,因为这将让 \TeX\ 进入垂直模式,从而结束该段落。
在 \xrefpg{filbreak} 中有此主题的更多建议。
\enddesc

\begindesc
\cts raggedbottom {}
\cts normalbottom {}
\explain
%Normally \TeX\ tries hard to ensure that all pages have the same depth,
%i.e., that their bottom margins are equal.
%The |\raggedbottom| command tells \TeX\ to
%allow some variability among the bottom margins on different pages.
%It's often appropriate to use |\raggedbottom| when you have material that
%contains large blocks of material that should not be split across pages.
%The |\normalbottom| command cancels the effect of |\raggedbottom|.
%\eix^^{page breaks//encouraging or discouraging}
通常 \TeX\ 会尽全力让每页都一样高,即让它们的下边距相等。
|\raggedbottom| 命令允许 \TeX\ 给不同页面的下边距一些差别。
如果文档中有些不能跨页的大型内容,用 |\raggedbottom| 就比较合适。
用 |\normalbottom| 命令可以取消 |\raggedbottom| 的效果.
\eix^^{page breaks//encouraging or discouraging}
\enddesc

%==========================================================================
%\subsection {Page breaking parameters}
\subsection{分页参数}

\bix^^{page breaks//parameters for}
\begindesc
\cts interlinepenalty {\param{number}}
\explain
%This parameter specifies the \minref{penalty} for
%breaking a page between the lines of a paragraph.
%By setting it to $10000$ you can force all page breaks to occur
%between paragraphs, provided that the pages have enough stretch so
%that \TeX\ can still compose them decently.
%\PlainTeX\ leaves |\inter!-linepenalty| at~$0$.
此参数设定在段落内部各行间分页时的\minref{惩罚}值。
将它设为 $10000$ 将强制在段落之间分页,
只要页面中有足够的伸展值让 \TeX\ 排版。
\PlainTeX\ 设定 |\inter!-linepenalty| 为 ~$0$。
\enddesc

\begindesc
\cts clubpenalty {\param{number}}
\explain
%^^{club line}
%This parameter specifies the \minref{penalty} for
%breaking a page just after the first line of a paragraph.
%A line by itself at the bottom of a page is called a ``club line''.
%\PlainTeX\ sets |\clubpenalty| to $150$.
^^{孤行}
此参数设定在段落首行之后分页的\minref{惩罚}值。
页面底部的单独一行称为``孤行''。
\PlainTeX\ 设定 |\clubpenalty| 为 $150$。
\enddesc

\begindesc
\cts widowpenalty {\param{number}}
\explain
%^^{widow line}
%This parameter specifies the \minref{penalty} for
%breaking a page just before the last line of a paragraph.
%A line by itself at the top of a page is called a ``widow line''.
%\PlainTeX\ sets |\widowpenalty| to $150$.
^^{寡行}
此参数设定在段落尾行之前分页的\minref{惩罚}值。
页面顶部的单独一行称为``寡行''。
\PlainTeX\ 设定 |\widowpenalty| 为 $150$。
\enddesc

\begindesc
\cts displaywidowpenalty {\param{number}}
\explain
%^^{widow line}^^{math display}
%This parameter specifies the \minref{penalty} for
%breaking a page just before the last line of a partial paragraph
%that immediately precedes a math display.
%\PlainTeX\ sets |\displaywidowpenalty| to $50$.
^^{寡行}^^{展示公式}
此参数设定在展示公式上一行同时也是段落尾行之前分页的\minref{惩罚}值。
\PlainTeX\ 设定 |\displaywidowpenalty| 为 $50$.
\enddesc

\begindesc
\cts predisplaypenalty {\param{number}}
\explain
%^^{math display}
%This parameter specifies the \minref{penalty} for
%breaking a page just before a math display.
%\PlainTeX\ sets |\predisplaypenalty| to $10000$.
^^{展示公式}
此参数设定在展示公式之前分页的\minref{惩罚}值。
\PlainTeX\ 设定\break |\pre!-display!-penalty| 为 $10000$。
\enddesc

\begindesc
\cts postdisplaypenalty {\param{number}}
\explain
%^^{math display}
%This parameter specifies the \minref{penalty} for
%breaking a page just after a math display.
%\PlainTeX\ leaves |\postdisplaypenalty| at $0$.
^^{展示公式}
此参数设定在展示公式之后分页的\minref{惩罚}值。
\PlainTeX\ 设定\break |\postdisplaypenalty| 为 $0$。
\enddesc

\begindesc
\cts brokenpenalty {\param{number}}
\explain
%This parameter specifies the \minref{penalty} for
%breaking a page just after a line  that ends in
%a discretionary item (usually a hyphen).
%^^{hyphenation}
%|\brokenpenalty| applies to page breaking, while
%^|\hyphenpenalty| \ctsref{\hyphenpenalty} applies to line breaking.
%\PlainTeX\ sets |\brokenpenalty| to $100$.
此参数设定在以自定连字项(通常是一个连字符)结尾的一行之后分页的\minref{惩罚}值。
^^{连字}
|\brokenpenalty| 用于分页, 而 ^|\hyphenpenalty| \ctsref{\hyphenpenalty} 用于断行。
\PlainTeX\ 设定 |\brokenpenalty| 为 $100$.
\enddesc

\begindesc
\cts insertpenalties {\param{number}}
\explain
%\bix^^{insertions//penalties for}
%\bix^^{floating material}
%This parameter contains the sum of certain penalties \minrefs{penalty}
%that \TeX\ accumulates as it is placing insertions onto the
%current page.
%These penalties are incurred when \TeX\ is processing an |\insert|
%command
%and discovers that a previous insertion  of the same kind on this page has
%been split, leaving part of it for subsequent pages.
%See \knuth{pages~123--125} for the details of this
%calculation.
%
%\margin{Material moved from a footnote to the main text.}
%|\insertpenalties| has an entirely different meaning during an
%^^{output routine//meaning of \b\tt\\insertpenalties\e\ in}
%output routine---it's the number of insertions that have been seen
%but that don't fit on the current page (see \knuth{page~125}).
\bix^^{insertions//penalties for}
\bix^^{floating material}
此参数包含 \TeX\ 放置插入项到当前页面时所累积的某些\minref{惩罚}值之和。
在 \TeX\ 处理 |\insert| 命令时,
如果发现前面有同类型插入项被分开,留下一部分到之后的页面,
就会算上这些惩罚值。
详细的计算过程可以见 \knuth{第~123--125~页}。

\margin{Material moved from a footnote to the main text.}
在输出例行程序中,
^^{output routine//meaning of \b\tt\\insertpenalties\e\ in}
|\insertpenalties| 有完全不同的意义——
它表示已经见到但未能放入当前页的插入项的总数(见 \knuth{第~125~页})。
\enddesc

\begindesc
\cts floatingpenalty {\param{number}}
\explain
%This parameter specifies the \minref{penalty} that
%\TeX\ adds to |\insertpenalties|
%^^|\insertpenalties|
%when the page builder is adding an insertion to the current page
%and discovers that a previous insertion  of the same kind on this page has
%been split, leaving part of it for subsequent pages.
%\PlainTeX\ leaves |\floatingpenalty| at~$0$.
%\eix^^{insertions//penalties for}
%\eix^^{floating material}
在 \TeX\ 处理插入项时,
如果发现前面有同类型插入项被分开,留下一部分到之后的页面,
就会添加 |\floatingpenalty| \minref{惩罚}值到|\insertpenalties|
^^|\insertpenalties|。
\PlainTeX\ 设定 |\floatingpenalty| 为 $0$。
\eix^^{insertions//penalties for}
\eix^^{floating material}
\enddesc

\begindesc
\cts pagegoal {\param{dimen}}
\explain
%This parameter specifies the
%desired height for the current page.
%\TeX\  sets
%|\pagegoal| to the current value of |\vsize|
%when it first puts a box or an insertion on the current page.
%You can shorten a page while \TeX\ is working on it by
%changing the value of |\pagegoal|---even if the new value is
%less than the height of the material already on that page.
%\TeX\ will just put the extra material on the next page.
%But remember---|\pagegoal| is reset to |\vsize| again when \TeX\
%starts the next page.
此参数设定当前页面所需的高度。
当放入第一个盒子或者插入项到当前页面时,
\TeX\ 设定 |\pagegoal| 等于 |\vsize| 的当前值。
改变 |\pagegoal| 的值可以让 \TeX\ 排版时缩短该页面的高度。
即使新的值比页面已有内容的高度还小也可以,
此时 \TeX\ 将把多余的内容放在下一页。
但记住——在下一页 \TeX\ 还是会重置 |\pagegoal| 为 |\vsize|。
\enddesc

\begindesc
\cts pagetotal {\param{dimen}}
\explain
%This parameter specifies the
%accumulated natural height of the current page.
%\TeX\ updates |\pagetotal| as it adds
%items to the main \minref{vertical list}.
此参数表示当前页面累积的自然高度。
\TeX\ 添加内容到\minref{垂直列}时会更新 |\pagetotal| 的值。
\enddesc

\begindesc
\cts pagedepth {\param{dimen}}
\explain
%This parameter specifies the depth of the current page.
%\TeX\ updates |\pagedepth| as it adds
%items to the main \minref{vertical list}.
此参数表示当前页面的深度。
\TeX\ 添加内容到\minref{垂直列}时会更新 |\pagedepth| 的值。
\enddesc

\begindesc
\cts pageshrink {\param{dimen}}
\explain
%This parameter specifies the amount of \minref{shrink}
%in the accumulated \minref{glue} on the current page.
%\TeX\ updates |\pageshrink| as it adds
%items to the main \minref{vertical list}.
此参数表示当前页面累积的\zglue 的可收缩量。
\TeX\ 添加内容到\minref{垂直列}时会更新 |\pageshrink| 的值。
\enddesc

\begindesc
\cts pagestretch {\param{dimen}}
\cts pagefilstretch {\param{dimen}}
\cts pagefillstretch {\param{dimen}}
\cts pagefilllstretch {\param{dimen}}
\explain
%These four parameters together specify the amount of \minref{stretch}
%in the \minref{glue} on the current page.  The amount of stretch has
%the form
%{\def\f#1{\hbox{\tt fi#1}}%
%$n_0 + n_1\f{l} + n_2\f{ll} + n_3\f{lll}$}, with the four parameters
%giving the values of the four $n_i$.
%\TeX\ updates these parameters as it adds
%items to the main \minref{vertical list}.
%
%\eix^^{page breaks//parameters for}
%\eix^^{page breaks}
这四个参数合起来表示当前页面累积的\zglue 的可伸长量。可伸长量总共等于
{\def\f#1{\hbox{\tt fi#1}}
$n_0 + n_1\f{l} + n_2\f{ll} + n_3\f{lll}$},
其中四个参数给出四个$n_i$的值。
\TeX\ 添加内容到\minref{垂直列}时会更新这些参数的值。
\eix^^{page breaks//parameters for}
\eix^^{page breaks}
\enddesc

%==========================================================================
%\section{Page layout}
\section{页面布局}

%\subsection {Page description parameters}
\subsection {页面描述参数}

\begindesc
\bix^^{page dimensions}
\margin{This command was also described in the `Paragraphs' section.
The combined description now appears there.}
\aux\cts hsize {\param{dimen}}
\explain
%This parameter specifies the current line length.
%See \xrefpg{\hsize} for a more complete explanation.
此参数指定当前行的长度。\PlainTeX\ 设定 |\hsize| 为 |6.5in|。
在\xrefpg{\hsize} 有完整的解释。
\enddesc

\begindesc
\cts vsize {\param{dimen}}
\explain
%This parameter specifies the current vertical extent of a page.
%\TeX\ examines it only when it is starting a page.  Thus if you change
%|\vsize| in the
%middle of a page, your change won't affect anything until the following page.
%If you want to change the vertical extent of a page when
%you're in the middle of it, you  should assign the new
%height to ^|\pagegoal| (\xref \pagegoal) instead.  (If you
%want the change to affect the following pages too, you should
%change \emph{both} |\vsize| and |\pagegoal|.)
%\PlainTeX\ sets |\vsize| to |8.9in|.
%\eix^^{page dimensions}
此参数指定页面的竖直长度。
\TeX\ 仅在页面开始时检查此值。
因此,在页面中间改变 |\vsize| 到下一页才会生效。
若要在页面中间改变其竖直长度,应该修改 ^|\pagegoal|(\xref \pagegoal )的值。%
(若要改变当前页面开始的所有页面,你需要\emph{同时}修改 |\vsize| 和 |\pagegoal|。)
\PlainTeX\ 设定 |\vsize| 为 |8.9in|。
\eix^^{page dimensions}
\enddesc

\begindesc
\cts hoffset {\param{dimen}}
\cts voffset {\param{dimen}}
\margin{These commands were moved up from later in the subsection.}
^^{page origin}
\idxsee{origin}{page origin}
\explain
%\TeX\ normally takes the ``origin'' of a page, that is, the point where
%it starts printing, as being one inch down from the top of the page and one
%inch to the right of the left end of the page.\footnote{
%\TeX\ itself is indifferent to where the origin of the page is,
%but this information has to be built into the device drivers
%^^{device drivers//page origin known to}
%that convert \dvifile s into printable form so that different devices
%will yield the same results.}
%The values of |\hoffset| and
%|\voffset| give the horizontal and vertical offset of the actual origin
%from this point.  Thus if |\hoffset| and |\voffset| are both
%zero, \TeX\ uses its normal origin.
\TeX\ 通常设定页面的“起点”——即它开始打印的地方——在离上页边和左页边均为一英寸的地方。\footnote{
\TeX\ 本身并不关心页面起点的位置,
但设备驱动程序将 \dvifile 转换为可打印形式时要用到,
此位置信息使得在不同的设备中能得到相同的结果。
^^{device drivers//page origin known to}}
|\hoffset| 和 |\voffset| 的值分别给出实际起点相对该点的水平和竖直偏移量。
因此,若 |\hoffset| 和 |\voffset| 均为零,\TeX\ 就使用正常起点。
\example
\hoffset = -.3in
   % Start printing .7 inches from left edge of paper.
\voffset = 1in
   % Start printing 2 inches from top edge of paper.
|
\endexample
\enddesc

\begindesc
\cts topskip {\param{glue}}
\explain
%\TeX\ inserts glue at the top of each
%page in order to ensure that the baseline of the first box on the page
%always is the same distance $d$ from the top of the page.
%|\topskip| determines the amount of that glue,
%called the ``|\topskip| glue'', by specifying
%what $d$ should be (provided that the first box
%on the page isn't too tall).
%$d$ is given by the natural size of the |\topskip| glue.
%If the height of the first box on the page exceeds $d$,
%so that the glue would be negative, \TeX\ simply inserts no
%|\topskip| glue at all on that page.
为确保页面第一个盒子的基线到上页边的距离总是 $d$,
\TeX\ 在每个页面顶部都插入\zglue 。
|\topskip| 给出了 $d$ 的值,以此确定该\zglue 大小
(只要页面第一个盒子不会太高)。
$d$ 等于 |\topskip| \zglue 的自然大小。
如果页面第一个盒子的高度超过 $d$,
计算出的粘连将为负值,此时在该页顶部 \TeX\ 不插入|\topskip|\zglue 。

%To understand better the effect of these rules, assume that |\topskip|
%has no stretch or shrink and that the first item on the page is indeed a box.
%Then if the height of that box is no greater than |\topskip|,
%its baseline will be |\topskip|
%from the top of the page independently of its height.  On the other hand,
%if the height of the box is $e$ greater than |\topskip|, its baseline will be
%|\topskip|\tplus$e$ from the top of the page.
%See \knuth{pages~113--114} for the remaining details of how
%|\topskip| works.
%\PlainTeX\ sets |\topskip| to |10pt|.
为更好地理解这些规则,我们假设 |\topskip| 无伸长量也无收缩量,且页面第一项就是个盒子。
如果该盒子的高度不大于 |\topskip|,不管高度为多少,
它的基线和上页边的距离就始终等于 |\topskip|。
反过来,如果该盒子的高度比 |\topskip| 大 $e$,
它的基线和上页边的距离就是 |\topskip|\tplus$e$。
在 \knuth{第~113--114~页} 中详细解释了 |\topskip| 的作用。
\PlainTeX\ 设定 |\topskip| 为 |10pt|。
\enddesc

\begindesc
\cts parskip {\param{glue}}
\explain
%This parameter specifies the ``^{paragraph skip}'', i.e., the
%vertical glue that \TeX\ inserts at the start of a paragraph.
%^^{paragraphs//glue between}
%See |\par| (\xref{\@par}) for more information about what happens
%when \TeX\ starts a paragraph.
%\PlainTeX\ sets |\parskip| to |0pt plus 0.1pt|.
此参数设定“^{段落间距}”,即 \TeX\ 在段落开头插入的竖直\zglue 。
^^{paragraphs//glue between}
在 |\par| (\xref{\@par}) 那里介绍了 \TeX\ 开始新段落时所作的事情。
\PlainTeX\ 设定 |\parskip| 为 |0pt plus 0.1pt|。
\enddesc

\begindesc
\cts maxdepth {\param{dimen}}
\explain
%This parameter specifies the maximum depth of the bottom \minref{box} on
%a page.  It is related to |\boxmaxdepth| (\xref \boxmaxdepth).  If the
%depth of the bottom box on a page exceeds |\maxdepth|, \TeX\ moves the
%box's reference point down so that it's
%|\maxdepth| from the bottom of that box.
%Without this adjustment, the bottom box on
%a page could extend well into the bottom margin or even off
%the page entirely.  \PlainTeX\ sets |\maxdepth| to |4pt|.
此参数设定页面底部\minref{盒子}的最大深度。
它和 |\boxmaxdepth| 参数(\xref \boxmaxdepth )也有关系.
如果底部盒子的深度超过 |\maxdepth|,
\TeX\ 下移该盒子的基准点,让深度就等于 |\maxdepth|。
若不作此调整,页面底部盒子可能大大超出下边距,甚至超出页面,
PlainTeX\ 设定 |\maxdepth| 为 |4pt|。
\enddesc

%==========================================================================
%\subsection{Page numbers}
\subsection{页码}

\begindesc
\bix^^{page numbering}
\cts pageno {\param{number}}
\explain
%This parameter contains the current page number as an integer.  The page
%number is normally negative for front-matter pages that are numbered
%with small roman numerals instead of arabic numerals.  If you change the
%page number within a page,
%the changed number will be used in any headers or footers that
%appear on that page.
%The actual
%printing of page numbers is handled by \TeX's \minref{output routine},
%which you can modify.
此参数以整数值记录当前页码。
在前页(front matter)中页码通常是负数,我们用小写罗马数字而不用阿拉伯数字编号。
在页面中间对页码的改动,会在该页的页眉和页脚中用到。
页码事实上由 \TeX\ 的\minref{输出例行程序}打印,你可以修改它们。

%\PlainTeX\ keeps the page number in the \minref{register} ^|\count0|.
%(|\pageno| is, in fact, a synonym for |\count0|.)
%Whenever it ships out a page to the \dvifile,
%^^|\shipout//{\tt\\count} registers displayed at|
%\TeX\ displays the current value of |\count0| on your
%terminal so that you can tell which page it is working on.
%It's possible to use registers |\count1|--|\count9| for nested
%levels of page numbers (you must program this yourself).
%If any of these registers are nonzero, \TeX\ displays them on your
%terminal also.\footnote{
%More precisely, it displays all registers in sequence from
%|\count0| to |\count9|, but omits trailing zero registers.
%For instance, if the values of |\count0|--|\count3|
%are $(17, 0 , 0, 7)$ and the others are $0$,
%\TeX\ displays the page number as {\tt [17.0.0.7]}.}
\PlainTeX\ 将页码记录在\minref{寄存器} ^|\count0| 中。%
(|\pageno| 实际上就是|\count0|的别名。)
当它输出一个页面到 \dvifile 时,
^^|\shipout//{\tt\\count} registers displayed at|
\TeX\ 在终端中显示  |\count0| 的值,让你知道它在排版哪一页。
也可以把寄存器 |\count1|--|\count9| 用于嵌套的多级页码
(但你必须自己编程)。
如果这些寄存器的值非零,\TeX\ 在终端中也显示它们的值。\footnote{
更准确来说,它从 |\count0| 到 |\count9| 依次显示各寄存器的值,但忽略末尾的零值。
比如,若 |\count0| 到 |\count3| 为 $(17, 0 , 0, 7)$ 而其他值都为 $0$,
则 \TeX\ 显示页码为 {\tt [17.0.0.7]}。}
\example
This explanation appears on page \number\pageno\
of our book.
|
\produces
This explanation appears on page \number\pageno\
of our book.

\nextexample
\pageno = 30 % Number the next page as 30.
Don't look for this explanation on page \number\pageno.
|
\produces
Don't look for this explanation on page 30.
\endexample
\enddesc

\begindesc
\cts advancepageno {}
\explain
%This command adds $1$ to the page number
%$n$ in |\pageno| if $n\ge0$
%and subtracts $1$ from it if $n<0$.
如果页码 $n\ge0$,此命令给 $n$ 加上 $1$;
如果 $n<0$ 则给 $n$ 减去 $1$。
\enddesc

\begindesc
\easy\cts nopagenumbers {}
\explain
%By default, \plainTeX\ produces a footer containing a centered page number.
%This command tells \TeX\ to produce a blank footer instead.
默认情况下,\plainTeX\ 会在页面的页脚生成居中的页码。
此命令让 \TeX\ 生成空的页脚.
\enddesc

\begindesc
\cts folio {}
\explain
%This command produces the current page number, whose value is the number
%$n$ contained in ^|\pageno|.
%If  $n\ge0$,
%\TeX\ produces $n$ as a decimal number,
%while if $n<0$,
%\TeX\ produces $-n$ in lowercase roman numerals.
此命令生成当前页码,它的值等于记录在 ^|\pageno| 中的 $n$。
若 $n\ge0$,\TeX\ 用十进制数形式生成 $n$;
若 $n<0$,\TeX\ 用小写罗马数字形式生成 $-n$。
\example
This explanation appears on page \folio\ of the book.
|
\produces
This explanation appears on page \folio\ of the book.
\endexample

\eix^^{page numbering}
\enddesc


%==========================================================================
%\subsection{Header and footer lines}
\subsection{页眉和页脚}

\begindesc\secondprinting{\vglue-.75\baselineskip\vskip0pt}
\cts headline {\param{token list}}
\cts footline {\param{token list}}
\explain
%These parameters
%contain, respectively, the current headline (header) and the current
%footline (footer).
%The \plainTeX\ output routine
%places the headline at the top of each page and the footline
%at the bottom of each page.
%The default headline is empty and the default footline is a centered
%page number.
这两个参数分别指定当前的页眉和页脚。
\PlainTeX\ 输出例行程序将把页眉放在各页顶部,页脚放在各页底部。
默认的页眉为空,而默认的页脚为居中的页码。

%The headline and footline should both
%be as wide as |\hsize| (use |\hfil|, \xref{\hfil}, for this
%if necessary).
%You should always include a font-setting command in these lines, since
%the current font is unpredictable when \TeX\ is calling the
%output routine.  If you don't set the font explicitly,
%you'll get whatever font \TeX\ was using when it broke the page.
页眉和页脚的宽度应该和 |\hsize| 相等(必要时可以用 |\hfil|,\xref{\hfil})。
在页眉和页脚中你永远应该加上字体设定命令,
因为你无法预知 \TeX\ 调用输出例行程序时的当前字体。
如果你不明确指明字体,\TeX\ 将使用分页时所用的字体。

%You shouldn't try to use |\headline| or |\footline|
%to produce multiline headers or footers.
%Although \TeX\ won't complain, it will give you something that's very ugly.
%See \xrefpg{bighead} for a method of creating multiline headers or
%footers.
不要试图用 |\headline| 或 |\footline| 生成多行的页眉或页脚。
即使 \TeX\ 不会报错,它也只会出很难看的结果。
在\xrefpg{bighead}中有生成多行页眉或页脚的方法。
\example
\headline = {\tenrm My First Reader\hfil Page \folio}
|
\produces
\pageno = 10
\line{\tenrm \noindent My First Reader\hfil Page \folio}
\par ({\it at the top of page \folio}\/)
\nextexample
\footline = {\tenit\ifodd\pageno\hfil\folio
           \else\folio\hfil\fi}
% Produce the page number in ten-point italic at
% the outside bottom corner of each page.
|
\endexample\enddesc

\secondprinting{\vfill\eject}


%==========================================================================
%\subsection {Marks}
\subsection {标记}

\begindesc
\cts mark {\rqbraces{\<text>}}
\explain
%\bix^^{marks}
%This command causes \TeX\ to append a mark
%containing \<mark text>
%to whatever list it is currently constructing.
%Generally you shouldn't use |\mark| within an
%``inner'' construct such as a math formula or a \minref{box}
%you've built with an |\hbox|, |\vbox|, or |\vtop| command, because
%\TeX\ won't see the mark when it's constructing the main box of the page.  But
%if you use |\mark| in ordinary horizontal \minref{mode}
%or directly in an hbox that's part of the main vertical list,
%the mark migrates out
%to the main \minref{vertical list}.
%See \knuth{pages~259--260} for examples showing how |\mark| can be used.
\bix^^{标记}
此命令让 \TeX\ 添加内容为 \<mark text> 的标记到当前正构造的列表中。
一般地,此标记命令不应该在 “内部”构造中使用,
例如数学公式或者用 |\hbox|、|\vbox| 或 |\vtop| 构造的\minref{盒子}。
这是因为 \TeX\ 构造页面主盒子时将看不到这样的标记。
但是,若你在普通水平\minref{模式}中用 |\mark|,
或者在主垂直列的 hbox 中直接使用 |\mark|,
该标记将会被转移到主\minref{垂直列}。
在\knuth{第~259--260~页}中有如何使用 |\mark| 的例子。
\enddesc

\begindesc
\cts firstmark {}
\cts botmark {}
\cts topmark {}
\explain
%These commands expand to the mark text in an item generated by an earlier
%^|\mark| command.
%The mark text has the form of a token list.
%\TeX\ sets the values of these commands when it
%finishes placing the contents of a page into ^|\box255|,
%just before calling the \minref{output routine}
%as part of its \minref{page break}ing actions.  \TeX\ determines
%these values as follows:
%\ulist\compact
%\li |\firstmark| contains the tokens of the first mark on the page.
%\li |\botmark| contains the tokens of the last mark on the page.
%\li |\topmark| contains the tokens of the mark that is in effect at
%the very top of the page.
%That mark is the last mark
%that \emph{preceded} the page,
%i.e., the |\botmark| of the previous page.
%It is empty if no marks preceded the page.
%\endulist
%\noindent
%If a page has no marks on it, \TeX\ will set |\firstmark| and |\botmark| to
%the same mark as |\topmark|, i.e., the most recent preceding mark.
%The table at the bottom of \knuth{page~258} illustrates the relation among
%|\firstmark|, |\botmark|, and |\topmark|.
这些命令展开为之前用 ^|\mark| 生成的标记项的标记文本。
标记文本以记号列形式表示。
执行分页工作时,\TeX\ 需要将页面内容放入 ^|\box255|,
然后调用\minref{输出例行程序};
在这两个步骤之间 \TeX\ 设定这些命令的值如下:
\ulist\compact
\li |\firstmark| 包含页面第一个标记的记号列。
\li |\botmark| 包含页面最后一个标记的记号列。
\li |\topmark| 包含紧跟在页面上边的那个标记的记号列。
这个标记就是\emph{上一个}页面的最后一个标记,即上一页的 |\botmark|。
如果上一页没有标记此命令为空。
\endulist
\noindent
如果此页面没有任何标记,\TeX\ 设定 |\firstmark| 和 |\botmark| 都等于 |\topmark|,
即前面最近的那个标记。
\knuth{第~258~页}底部的表格说明了 |\firstmark|、|\botmark| 和 |\topmark| 这三者的关系。
\enddesc

\begindesc
\cts splitfirstmark {}
\cts splitbotmark {}
\explain
%^^{marks//for split lists}
%These commands expand to the ^{mark text} generated by an earlier
%^|\mark| command that produced an item in the item list of a vbox $V$.
%The mark text has the form of a token list.
%When \TeX\
%splits $V$ in response to a ^|\vsplit| command (\xref \vsplit),
%it sets the values of these commands as follows:
%\ulist\compact
%\li |\splitfirstmark| contains the tokens of the first mark in
%the item list of $V$.
%\li |\splitbotmark| contains the tokens of the last mark in
%the item list of $V$.
%\endulist
%These commands produce no tokens if there was no preceding |\vsplit|,
%or if the most recent preceding |\vsplit| didn't contain any marks.
%\eix^^{marks}
^^{marks//for split lists}
这两个命令展开为之前用 ^|\mark| 在 vbox $V$ 中生成的标记项的标记文本。
标记文本以记号列形式表示。
当你用 ^|\vsplit| 命令(\xref \vsplit )分割 $V$ 时,
\TeX\ 按如下方式设定这两个命令的值:
\ulist\compact
\li |\splitfirstmark| 包含 $V$ 中第一个标记的记号列。
\li |\splitbotmark| 包含 $V$ 中最后一个标记的记号列。
\endulist
如果没有 |\vsplit| 命令,或者前面最近的 |\vsplit| 不包含任何标记,
这两个命令不生成记号列。
\eix^^{标记}
\enddesc

%==========================================================================
%\section {Insertions}
\section {插入项}

%\subsection {Footnotes}
\subsection {脚注}

\begindesc
\easy\cts footnote {\<argument$_1$> \<argument$_2$>}
\cts vfootnote {\<argument$_1$> \<argument$_2$>}
\explain
%These commands produce footnotes.
%\<argument$_1$> is the ``^{reference mark}''
%for the footnote and \<argument$_2$> is its text.
%The text can be several paragraphs long if necessary and can contain
%constructs such as math displays, but it shouldn't contain any
%\minref{insertion}s (such as other footnotes).
这两个命令用于生成脚注。
其中 \<argument$_1$> 是脚注的“^{参考符号}”,
而 \<argument$_2$> 是脚注文本。
脚注文本必要时可以包含多个段落,也可以包含展示公式等其他构造,
但不应该包含任何\minref{插入项}(比如其他脚注)。

%You shouldn't use these commands inside a subformula of a math formula,
%in a box within a box being contributed to a page,
%or in an insertion of any kind.
%If you're unsure whether these restrictions apply, you can be safe by
%only using |\footnote| and |\vfootnote| directly within a paragraph or
%between paragraphs.
在数学公式的子公式中,或者组成页面的盒子的子盒子中,
或者任何形式的插入项中,你都不应该使用这些命令。
如果你不确定这些限制出现在哪里,为安全起见,
你可以只在段落内或者段落间直接使用 |\footnote| 和 |\vfootnote|。

%These restrictions aren't as severe as they seem because you can use
%|\vfootnote| to footnote most anything.
%Both |\foot!-note| and |\vfoot!-note| insert the reference mark in front of the
%footnote itself, but |\vfoot!-note| doesn't insert the reference mark into the
%text.
%Thus, when you use |\vfoot!-note| you can
%explicitly insert the reference mark
%wherever it belongs without concern about the context
%and place the |\vfootnote| in the next paragraph.
%If you find that the footnote lands on
%the page following the one where it belongs, move the |\vfootnote| back
%to the previous paragraph.
%There are rare circumstances where you'll need to
%alter the text of your document in order to get a footnote to appear on
%the same page as its reference mark.
这些限制没看起来那么严重,因为你可以用 |\vfootnote| 命令添加几乎任何内容的脚注。
|\foot!-note| 和 |\vfoot!-note| 都会在脚注前面插入参考符号,
区别在于 |\vfoot!-note| 不会在正文中插入参考符号。
因此,你可以不用考虑上下文的限制,
显式地将参考符号放到它应该在的地方,然后在下一个段落使用 |\vfootnote| 命令。
这可能会导致脚注跑到下一页去,此时你可以将 |\vfootnote| 命令移动到上一个段落中。
在极少情况下,你需要修改文档正文,以让脚注和参考符号出现在同一页中。
\example
To quote the mathematician P\'olya is a ploy.\footnote
*{This is an example of an anagram, but not a strict one.}
|
\produces
To quote the mathematician P\'olya is a ploy.*
\par\line{\hskip .5in \vdots\hfil}
\nointerlineskip \bigskip
\footnoterule\par\parindent = 12pt
\textindent{*}This is
an example of an anagram, but not a strict one.
\endexample
\example
$$f(t)=\sigma\sigma t\;\raise 1ex \hbox{\dag}$$
\vfootnote \dag{The $\sigma\sigma$ notation was explained in
the previous section.}
|
\produces
$$f(t)=\sigma\sigma t\>\raise 1ex \hbox{\dag}$$
\par\line{\hskip .5in \vdots\hfil}
\nointerlineskip \bigskip
\footnoterule\par\parindent = 12pt
\textindent{\dag}{The $\sigma\sigma$ notation was explained in
the previous section.}
\endexample
\enddesc

%==========================================================================
%\subsection {General insertions}
\subsection {一般插入项}

\begindesc
\bix^^{insertions//commands for}
\cts topinsert {\<vertical mode material> {\bt \\endinsert}}
\cts midinsert {\<vertical mode material> {\bt \\endinsert}}
\cts pageinsert {\<vertical mode material> {\bt \\endinsert}}
\explain
%These commands produce different forms of insertions that
%instruct
%(or allow) \TeX\ to relocate the \<vertical mode material>:
%\ulist
%\li |\topinsert| attempts to put the material at the top of the current page.
%If it won't fit there, |\topinsert|
%will move the material to the next available top of page.
%\li |\midinsert| attempts to put the material at the current position.
%If it won't fit there, |\midinsert|
%will move the material to the next available top of page.
%\li |\pageinsert| puts the material by itself on the next page.
%To avoid an underfull page, be sure to end the inserted material with
%|\vfil| or fill out the excess space some other way.
%% Knuth doesn't say this, but I tried an experiment that verified it.
%% Nor does he say explicitly that an insertion does a \par.
%\endulist
%\noindent
%The \<vertical mode material>
%is said to be ``floating'' ^^{floating material} because \TeX\
%can move it from one place to another.
%Insertions are very useful for material such as figures and tables because
%you can position such material where you want it without knowing where the
%page breaks will fall.
这些命令生成不同形式的插入项,
它们指示(或者允许)\TeX\ 重新放置 \<vertical mode material>:
\ulist
\li |\topinsert| 试图将内容放在当前页的顶部。
如果那里放不下,|\topinsert| 将把该内容移动到顶部可放下的下一个页面。
\li |\midinsert| 试图将内容放在当前位置,
如果当前位置放不下,|\midinsert| 将把该内容移动到顶部可放下的下一个页面。
\li |\pageinsert| 将把内容单独放在下一页。
为免得到松散页面,不要忘记在内容最后用 |\vfil| 命令或者其他方式填满页面。
% Knuth doesn't say this, but I tried an experiment that verified it.
% Nor does he say explicitly that an insertion does a \par.
\endulist
\noindent
由于 \TeX\ 可以移动\<vertical mode material>的位置,我们称它为“浮动体”。^^{浮动体}
插入项对于图形和表格非常有用,
因为你可以将它们放在所需的地方,而不用管哪里会分页。

%Each of these commands implicitly ends the current paragraph, so
%you should use them only between paragraphs.
%You should not use them within a box or within another insertion.
%If you have several insertions competing for
%the same space, \TeX\ will retain their relative order.
这些命令每个都会结束当前段落,因此你应该仅在段落之间使用它们。
你也不要在盒子内部或者另一个插入项内部使用它们。
如果有多个插入项要放在同一个位置,\TeX\ 将保持它们的先后顺序。
\example
\pageinsert
% This text will appear on the following page, by itself.
This page is reserved for a picture of the Queen of Hearts
sharing a plate of oysters with the Walrus and
the Carpenter.
\endinsert
|
\endexample
\enddesc

\begindesc
\cts endinsert {}
\explain
%This command ends an insertion started by
%|\topinsert|, |\midinsert|, or |\pageinsert|.
此命令结束由 |\topinsert|、|\midinsert| 或 |\pageinsert| 命令开始的插入项。
\enddesc

\begindesc
\cts insert {\<number> \rqbraces{\<vertical mode material>}}
\explain
%\minrefs{insertion}
%This primitive command provides the
%underlying mechanism for constructing insertions, but
%it is hardly ever used outside of a \minref{macro} definition.
%The definitions of the
%|\foot!-note|, |\vfoot!-note|,
%|\top!-insert|, |\mid!-insert|, and |\page!-insert| commands are
%all built around |\insert|.
%^^|\topinsert|
%^^|\midinsert| ^^|\pageinsert| ^^|\footnote| ^^|\vfootnote|
\minrefs{insertion}
这个原始命令提供了构造插入项的内部机制,
但除了在定义\minref{宏}时会用到,其他时候很少用到它。
|\foot!-note|、|\vfoot!-note|、|\top!-insert|、|\mid!-insert| 和
|\page!-insert| 命令的定义都建立在 |\insert| 命令之上。
^^|\topinsert|
^^|\midinsert| ^^|\pageinsert| ^^|\footnote| ^^|\vfootnote|

%When you design insertions for a document, you should assign a different
%integer code\footnote
%{\texbook{} uses the term ``class'' for a code.
%We use a different term to avoid confusion with the other meaning of
%``class'' (\xref{class}).}
%$n$ to each kind of insertion,
%using the ^|\newinsert| command (\xref{\@newinsert}) to obtain the
%integer codes.
%The |\insert| command itself appends the \<vertical mode material>
%to the current horizontal or \minref{vertical list}.
%Your \minref{output routine} is responsible for
%moving the inserted material from where it resides in |\box|$\,n$
%to an output page.
%^^{output routine}
设计文档的插入项时,不同的插入项需要分配不同的整数代码\footnote
{\texbook{} 用“类别”这个术语称呼此代码。
我们用不同的术语,以免和“类别”(\xref{class})的其他意思混淆。}
$n$,这个代码可以用 ^|\newinsert| 命令(\xref{\@newinsert})得到。
|\insert|命令本身仅添加 \<vertical mode material> 到当前水平列或\minref{垂直列}。
而\minref{输出例行程序}负责移动|\box|$\,n$中的插入内容到输出页面。
^^{输出例行程序}

%\TeX\ groups together all insertions having the same code
%number.  Each insertion
%code $n$ has four \minref{register}s associated with it:
%\ulist
%\li |\box|$\,n$ is where \TeX\ accumulates the material for insertions
%with code $n$.  When \TeX\ breaks a page, it puts into |\box|$\,n$
%as much insertion $n$ material as will fit on the page.
%Your output routine should then move this material to the actual page.
%You can use ^|\ifvoid| \ctsref{\@ifvoid}
%to test if there is any material in |\box|$\,n$.
%If not all the material fits, \TeX\ saves the leftovers for the next page.
%\li |\count|$\,n$ is a magnification factor $f.$  When \TeX\ is computing
%the vertical space occupied on the page
%by insertion $n$ material, it multiplies the
%vertical extent of this material by $f/1000$.
%Thus you would ordinarily set $f$ to $500$ for a double-column insertion
%and to $0$ for a marginal~note.
%\li |\dimen|$\,n$ specifies the maximum amount
%of insertion $n$ material that \TeX\ will put on a single page.
%\li |\skip|$\,n$ specifies extra space that \TeX\ allocates on the page
%if the page contains any insertion $n$ material.
%This space is in addition to the space occupied by the insertion itself.
%For example, it would account for the space on a page above the footnotes
%(if there are~any).
%\endulist
%\noindent
%\TeX\ sets |\box|$\,n$, and you should set the other three registers
%so that \TeX\ can correctly compute the vertical space required by the
%insertion.
%See \knuth{pages~122--125} for further details of how \TeX\ processes this
%command and of how insertions interact with page breaking.
%\xrdef{endofinsert}
\TeX\ 将所有插入项按照它们的代码分组。每个插入代码关联四个\minref{寄存器}:
\ulist
\li |\box|$\,n$ 给 \TeX\ 用于累积第 $n$ 种插入内容。
\TeX\ 分页时会尽可能多地往 |\box|$\,n$ 中放入第$n$ 种插入项。
然后输出例行程序将把这些内容移动了实际页面上。
你可以用 ^|\ifvoid| 命令\ctsref{\@ifvoid}检查 |\box|$\,n$ 中是否有任何内容。
如果不能放下全部内容,\TeX\ 将把剩下内容留到下一页。
\li |\count|$\,n$ 为放大因子 $f$。
\TeX\ 计算第$n$种插入项所占用的页面纵向空间时,将把插入内容的纵向长度乘以$f/1000$。
因此对双栏插入项通常设 $f$ 为 $500$,而对边注项设为 $0$。
\li |\dimen|$\,n$ 设定每页放置的第$n$种插入项的最大数目。
\li |\skip|$\,n$ 设定页面包含第 $n$ 种插入项时 \TeX\ 添加的额外间隔。
这个间隔不包含该插入项本身占用的空间。
例如,当页面包含脚注时,就会用到此间隔。
\endulist
\noindent
\TeX\ 只设定 |\box|$\,n$,你需要设定其他三个寄存器,
以让 \TeX\ 正确计算插入项所需的竖直空间。
\knuth{第~122--125~页} 中详细介绍了 \TeX\ 如何处理此命令,
以及插入项和分页方式之间的关系。
\xrdef{endofinsert}
\enddesc

\see |\floatingpenalty| (\xref \floatingpenalty).
\eix^^{insertions//commands for}


%==========================================================================
%\section{Modifying the output routine}
\section{修改输出例行程序}

%\bix^^{output routine}
%\begindesc
%\cts output {\param{token list}}
%\explain
%This parameter contains the current \minref{output routine}, i.e.,
%the token list that \TeX\ expands when it finds a page break.
%\TeX\ puts the page into ^|\box255|, so |\output| is responsible
%for doing something with |\box255|---either shipping it out or
%putting it somewhere else.
%The output routine is also responsible for attaching things such as
%headers and footers.
%\enddesc

\bix^^{输出例行程序}
\begindesc
\cts output {\param{token list}}
\explain
这个变量包含了当前的 \minref{输出例行程序},也就是 \TeX\ 当发现一个断页点的时候会展开的一个符号列表。
\TeX\ 把当前页面放入 ^|\box255|,所以 |\output| 负责对 |\box255| 进行处理——不管是把它输出到 dvi 还是把它放到别的什么地方去。
输出例行程序也负责把天头或者地脚附加到页面上。
\enddesc

%\begindesc
%\cts plainoutput {}
%\explain
%This command invokes \plainTeX's output routine.
%^^{output routine//default in \plainTeX}
%\PlainTeX\ defines ^|\output| as
%a token list containing the single token |\plainoutput|.
%\enddesc

\begindesc
\cts plainoutput {}
\explain
这个命令调用 \plainTeX\ 的输出例行程序。
^^{\plainTeX 的默认输出例行程序}
\PlainTeX\ 把 ^|\output| 定义为一个仅仅包含 |\plainoutput| 的符号列表。
\enddesc


%\begindesc
%\cts shipout {\<box>}
%\explain
%This command instructs \TeX\ to send \<box> to the \dvifile.
%^^{\dvifile//receives boxes from \b\tt\\shipout\e}
%\TeX\ expands any |\write| command in \<box> as part of |\shipout|.
%^^|\write//expanded during {\tt\\shipout}|
%The principal use of |\shipout| is in the output routine, but you can
%use it anywhere.
%\enddesc

\begindesc
\cts shipout {\<box>}
\explain
这个命令让 \TeX\ 把 \<box> 输出到 \dvifile.
^^{\dvifile//接收 \b\tt\\shipout\e 的盒子}
\TeX\ 在 |\shipout| 时,把 \<box> 中的任何 |\write| 命令进行展开。
^^|\write// 在 {\tt\\shipout} 时进行展开|
|\shipout| 的主要用途是在输出例行程序中,不过你可以在任何其他地方使用这个命令。
\enddesc


%\begindesc
%\cts deadcycles {\param{number}}
%\explain
%This parameter contains the number of times that \TeX\ has initiated
%the \minref{output routine} since the last time it did a
%^^|\shipout| |\ship!-out|.\footnote{
%More precisely, \TeX\ sets |\dead!-cyles| to $0$ whenever it executes
%|\ship!-out|
%and increments it by $1$ whenever it executes |\output|.}
%If |\deadcycles| gets too big, you've probably gotten
%\TeX\ into a loop, e.g., one where the page builder is trying the same page
%break over and over again.
%\enddesc

\begindesc
\cts deadcycles {\param{number}}
\explain
这个变量包含这 \TeX\ 在上次 ^^|\shipout| |\ship!-out| 后已经初始化的 \minref{输出例行程序} 的个数。
\footnote{更加准确地说, \TeX\ 在执行 |\ship!-out| 的时候,把 |\dead!-cycles| 归 $0$,然后每次执行 |\output| 的时候把它增加 $1$。}
如果 |\deadcycles| 变得太大,那你很有可能把 \TeX\ 弄死循环了,也就是说,循环往复地对同一个页面进行分页操作。
\enddesc


%\begindesc
%\cts maxdeadcycles {\param{number}}
%\explain
%If the value of |\deadcycles| exceeds the value of |\maxdeadcycles|,
%\TeX\ assumes that the output routine has gotten into a loop.
%\TeX\ then complains and runs its own simple output routine,
%equivalent to |\shipout!allowbreak\box255|,
%that is likely to break the loop.
%\PlainTeX\ sets |\maxdeadcycles| \hbox{to $25$}.
%\enddesc

\begindesc
\cts maxdeadcycles {\param{number}}
\explain
如果 |\deadcycles| 超过了 |\maxdeadcycles|,那 \TeX\ 就会假定输出例行程序被用户死循环了,
\TeX\ 这个时候就会报错,并且执行它自己默认的简单输出例行程序,也就是 |\shipout!allowbreak\box255|,
以期望从循环中跳出。\PlainTeX\ 把 |\maxdeadcycles| 设置为 \hbox{$25$}。
\enddesc

%\begindesc
%\cts outputpenalty {\param{number}}
%\explain
%\TeX\ sets this parameter when it breaks a page.  If the breakpoint was at a
%\minref{penalty} item, \TeX\ removes the penalty item and
%sets |\outputpenalty| to the penalty value at
%the breakpoint; otherwise it sets |\outputpenalty| \hbox{to $0$}.
%
%Suppose that you are undoing a page break
%in order to break the page at a different place than the one that \TeX\
%has just chosen.  In order to reconstruct the page, you need to recreate
%the penalty at \TeX's chosen breakpoint.
%You can accomplish this with the command
%|\penalty!allowbreak\outputpenalty|.
%\enddesc

\begindesc
\cts outputpenalty {\param{number}}
\explain
\TeX\ 在断页的时候设置这个参数。
如果断点是在一个 \minref{惩罚} 的项目上,\TeX\ 就会移除这个惩罚,
然后把 |\outputpenalty| 设置成断点处的值, 否则 |\outputpenalty| 就被设成 \hbox{$0$}。

如果你希望在 \TeX\ 选择的断点之外的的其他地方断页,而不在该断点处断页,
那么在重构这个页面时,你就需要在 \TeX\ 选择的断点处重新设置这个惩罚。
你可以通过 |\penalty!allowbreak\outputpenalty| 来达到这个目的。
\enddesc


%\begindesc
%\cts holdinginserts {\param{number}}
%\explain
%If this parameter is greater than $0$ when \TeX\
%is processing a page break, \TeX\ will refrain from processing insertions.
%Setting this parameter to $1$
%can be useful when you're writing an output routine
%that needs to reprocess the contents of the page, e.g.,
%an output routine that uses a value of |\vsize| (\xref \vsize) different from
%the one used by the page builder.
%
%\eix^^{output routine}
%\enddesc

\begindesc
\cts holdinginserts {\param{number}}
\explain

当 \TeX\ 在处理一个断页时,这个参数如果大于 $0$, 那么 \TeX\ 就会阻止处理插入。
当你写一个输出例行程序的时候,如果你希望重新处理该页的内容,
(例如输出例行程序本身用的 |\vsize| (\xref \vsize) 和断页程序用的并非相同值),
那么把这个参数设为 $1$ 就会很有用。

\eix^^{输出例行程序}
\enddesc


%==========================================================================
%\section {Splitting vertical lists}
\section {分割垂直列表}

%\begindesc
%\cts vsplit {\<number> {\bt to} \<dimen>}
%\explain
%This command causes \TeX\ to split the \minref{box} numbered
%\<number>, which we'll call $B_2$, into two parts.
%It uses the same algorithm that it would use if $B_2$ was a page
%and it was breaking that page;
%the division point then corresponds to the page break that it would find.
%The box $B_2$ must be a vbox, not an hbox.
%% we avoid starting the previous sentence with a symbol, a copyediting no-no.
%\TeX\ puts the material preceding the division point into
%another box $B_1$ and leaves the material after the division point in $B_2$.
%The |\vsplit| command then produces $B_1$.
%Normally you'd assign $B_1$ to a different
%box register, as in the example below.
%If the division point is at the end of $B_2$,
%$B_2$ will be empty after the |\vsplit|.

\begindesc
\cts vsplit {\<number> {\bt to} \<dimen>}
\explain
这个命令使得 \TeX\ 把寄存器中编号为 \<number> 的\minref{盒子} $B_2$ 切割成两个部分。
如果 $B_2$ 是一个页面,那这个命令会使用相同的算法来把这个页面分为两部分;
这种情况下,分割处就是断页的地方。
$B_2$这个盒子必须是个垂直盒子,不是水平盒子。
\TeX\ 把从开头到分割处的部分放到另一个盒子 $B_1$ 中,而把分割处到结尾的部分放到 $B_2$ 中。
然后 |\vsplit| 就输出了 $B_1$ 这个盒子。
一般情况下,你会用类似下面例子的方式,把 $B_1$ 盒子指定到另一个盒子寄存器中。
如果分割处恰好是 $B_2$ 盒子的结尾部分,那 $B_2$ 盒子就会在 |\vsplit| 后变成一个空盒子。


%\TeX\ employs its usual page-breaking algorithm
%^^{page breaks//in split lists}
%for the split.
%It uses \<dimen> for ^|\pagegoal|, the desired height of $B_1$.
%The vertical extent of $B_1$ may not be exactly
%\<dimen> because \TeX\ may not be able to achieve its page goal perfectly.
%\TeX\ does not consider insertions in calculating the split,
%so insertions in the original vertical list of $B_2$ will be retained
%but won't affect the split point.

\TeX\ 使用它一贯的断页算法来分割这个垂直盒子。
它使用 \<dimem> 的长度来指定 $B_1$ 盒子需要的高度 ^|\pagegoal|。
而$B_1$ 的垂直长度未必就一定是 \<dimem>,因为 \TeX\ 不一定能完美地达到其分页的目标。
\TeX\ 在计算这个垂直分割的时候,不考虑插入,因此 $B_2$ 插入的原本的垂直列表依然会被保留,但并不会受到分割的影响。


%\example
%\setbox 20 = \vsplit 30 to 7in
%% Split off the first seven inches or so of material from
%% box 30 and place that material in box 20.
%|
%\endexample
%\enddesc

\example
\setbox 20 = \vsplit 30 to 7in
% 把30号盒子截去7英寸,并放到20号盒子中。
|
\endexample
\enddesc


%\begindesc
%\cts splitmaxdepth {\param{dimen}}
%\explain
%This parameter specifies the
%maximum allowable depth of a box resulting from a |\vsplit|.
%|\splitmaxdepth| plays the
%same role that ^|\maxdepth| (\xref \maxdepth) plays for a page.
%\enddesc

\begindesc
\cts splitmaxdepth {\param{dimen}}
\explain
这个参数指定|\vsplit|产生的盒子的最大允许深度。
|\splitmaxdepth| 在盒子中和 ^|\maxdepth| (\xref \maxdepth) 在页面中扮演的角色相同。
\enddesc


%\begindesc
%\cts splittopskip {\param{glue}}
%\explain
%This parameter specifies the glue that \TeX\ inserts at the top of
%a box resulting from a |\vsplit|.
%|\splittopskip| plays the
%same role that ^|\topskip| (\xref \topskip) plays for a page.
%\enddesc

\begindesc
\cts splittopskip {\param{glue}}
\explain
这个参数指定 \TeX 在 |\vsplit| 产生的盒子的头部插入的粘连。
|\splittopskip| 在盒子中和 ^|\topskip| (\xref \topskip) 在页面中扮演的角色相同。
\enddesc


%\see |\splitbotmark|, |\splitfirstmark| (\xref \splitfirstmark).

\see |\splitbotmark|, |\splitfirstmark| (\xref \splitfirstmark).

\enddescriptions \endchapter \byebye
