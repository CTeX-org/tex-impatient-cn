~% -*- coding: utf-8 -*-
~% This is part of the book TeX for the Impatient.
~% Copyright (C) 2003 Paul W. Abrahams, Kathryn A. Hargreaves, Karl Berry.
~% See file fdl.tex for copying conditions.
~% TeX ignores anything on a line after a %
~% The next two lines define fonts for the title
% TeX 忽略一行中 % 之后的任何内容
% 下面两行定义标题所用的字体
\font\xmplbx = cmbx10 scaled \magstephalf
\font\xmplbxti = cmbxti10 scaled \magstephalf
~% Now here's the title.
% 现在是标题行
~%\leftline{\xmplbx Example !xmpnum:\quad\xmplbxti Entering simple text}
~\bookmark{2}{例!xmpnum:输入简单文本}%
\leftline{\xmplbx 例!xmpnum:\quad\xmplbxti 输入简单文本}
\vglue .5\baselineskip % skip an extra half line
~\count255 = \pageno
~\xdef\examplepage{\number\count255}
~\markinfo{Example 1: Entering simple text}
~\ifrewritetocfile
~\write\tocfile{\string\tocsectionentry{输入简单文本}{}{\examplepage}}%
~\fi
~^^{footnotes} ^^{comments} ^^{punctuation} ^^{quotation marks}
~^^{space characters} ^^{dashes} ^^{paragraphs//ending}
~\edef\examplepageno{\number\count255}%
~%It's easy to prepare ordinary text for \TeX\ since
~%\TeX\ usually doesn't care about how you break up your input into
~%lines. It treats the end of a line of text like a space.%
~%\footnote \dag{\TeX\ treats a tab like a space too, as we point
~%out in this {\it footnote}.}   If you don't want a space there,
~%put a per%
~%cent sign (the comment character) at the end of the line.
~%   \TeX\ ignores spaces at the start of a line, and treats more
~%than        one      space as equivalent to a single space,
~%even after a period.      You indicate a new paragraph by
~%skipping a line (or more than one line).

由于 \TeX\ 并不关心原始的输入文本中的换行,因此,我们无须关心
应该给其提供何种格式化的文本。在 \TeX\ 看来,输入文本中的换行符
和空格等价。%
\footnote \dag{\TeX\ 将 Tab 也当作空格处理, 就如你在这个
{\it 脚注}所见到的一样。}   如果你不想在行尾处看到空格,
可以在行尾处写一个百分号(注释符号)。\TeX\ 忽略行首的空格,
对多于一个的空格当做一个单独的空格看待 --- 哪怕这些空格位于
句号后面。你可以用一个或者多个空白行表示一个新的段落的开始。

~%When \TeX\ sees a period followed by a space (or the end of the
~%line, which is equivalent), it ordinarily assumes you've ended a
~%sentence and inserts a little extra space after the period.  It
~%treats question marks and exclamation points the same way.
当 \TeX\ 看见一个句号后面紧接着一个空格(或者是一个行结束符,这
两者等价)时,它通常会认为此句号代表一句话的结束,并在句号的后面
加上一点小小的空格。同样的处理还适用于问号和惊叹号。

~%  But \TeX's rules for handling periods sometimes need fine
~%tuning. \TeX\ assumes that a capital letter before a period
~%doesn't end the sentence, so you have to do something a little
~%different if, say, you're writing about DNA\null.
有时需要微调 \TeX 处理句号的方式。当句号前面紧跟着标题字
母(大写字母)时,\TeX\ 会认为这个句号并不做为句子的结束符,
因此在这种情况下,你需要作一些小改动。例如:我们这样写 DNA\null.
~% The \null prevents TeX from perceiving the capital `A'
~% as being next to the period.
% \null 可以阻止 \TeX\ 感知句号前面的标题字母 `A'。
~%It's a good idea to tie words together in references such as
~%``see Fig.~8'' and in names such as V.~I\null. Lenin and in
~%$\ldots$ so that \TeX\ will never split them in an awkward place
~%between two lines.  (The three dots indicate an ellipsis.)
~% 这个 references 不知道这样翻译对不对
如果想将多个单词绑定到一起,中间不分行,则可以将它们放入一个引用中,
例如:``见 Fig.~8'';或者名字中,例如:V.~I\null. Lenin;
或者省略号 $\ldots$。在这样的情况下,\TeX\ 不会在它们之间断行
(三个点表示省略号)。

~%You should put quotations in pairs of left and right
~%single ``quotes'' so that you get the correct left and right
~%double quotation marks.  ``For adjacent single and double
~%quotation marks, insert a `thinspace'\thinspace''. You can
~%get en-dashes--like so, and em-dashes---like so.
为得到正确的左右双引号,你需要用两个左右单``引号''。
``在相邻的单引号和双引号之间,要插入一个`小间隔'\thinspace''。
连接号--可以这样写,而破折号---可以这样写。

~%\bye % end the document
\bye % 结束此文档
:::
~%\xmpheader !xmpnum/{Indentation}% !xmpheaddef
\xmpheader !xmpnum/{缩进}% !xmpheaddef
~^^{indentation} ^^{margins} ^^{paragraphs//narrow}
~%\noindent Now let's see how to control indentation.  If an
~%ordinary word processor can do it, so surely can \TeX. Note
~%that this paragraph isn't indented.
\noindent 现在让我们来看看如何控制缩进。如果一个普通的文字处理
程序都能处理缩进的话,那 \TeX\ 一定也能。请注意,本段并未缩进。

~%Usually you'll either want to indent paragraphs or to leave
~%extra space between them.  Since we haven't changed anything
~%yet, this paragraph is indented.
通常来说,你会想在段落的起始处安排缩进,在段落之间安排一些额外
的空白。由于我们在此尚未做任何特殊处理,因此本段是缩进的。

{\parindent = 0pt \parskip = 6pt
~% The left brace starts a group containing the unindented text.
% 这里的左括弧包裹了一组不缩进的文本

~%Let's do these two paragraphs a different way,
~%with no indentation and six printer's points of extra space
~%between paragraphs.
让我们对这两个段落作一些特殊处理,这两段将不缩进,并且两段
之间有 6pt 宽的空白。

~%So here's another paragraph that we're typesetting without
~%indentation.  If we didn't put space between these paragraphs,
~%you would have a hard time knowing where one ends
~%and the next begins.
这是第二段不缩进的文字。如果不在这两段之间插入空白的话,将很难
分辨段与段之间的结束和起始。

\par % The paragraph *must* be ended within the group.
}% The right brace ends the group containing unindented text.

~%It's also possible to indent both sides of entire paragraphs.
~%The next three paragraphs illustrate this:
你也能在段落的两边同时缩进,下面的 3 段文字演示了双边缩进的效果:

\smallskip % Provide a little extra space here.
% Skips like this and \vskip below end a paragraph.
{\narrower
~% ``We've indented this paragraph on both sides by the paragraph
~% indentation.  This is often a good way to set long quotations.''
``我们对这一段作了双边缩进的处理。在长段引用时,通常都会这么做。''

~%``You can do multiple paragraphs this way if you choose.  This
~%is the second paragraph that's singly indented.''\par}
``如果你愿意的话,可以同时对多个段落作如此处理.  这是
第二个轻微缩进的段落.''\par}

~%{\narrower \narrower You can even make paragraphs doubly narrow
~%if that's what you need.  This is an example of a doubly
~%narrowed paragraph.\par}
{\narrower \narrower 如果需要,你也可以让段落缩进双倍长度。
这就是一个双倍缩进段落的例子.\par}
\vskip 1pc % Skip down one pica for visual separation.
~%In this paragraph we're back to the normal margins, as you can
~%see for yourself.  We'll let it run on a little longer so that
~%the margins are clearly visible.
在这一段,如你所见,我们返回到正常的边距状态。我们尽量让这一段
稍微长一点,这样你可以更清楚地看到边距的效果。

~%{\leftskip .5in Now we'll indent the left margin by half
~%an inch and leave the right margin at its usual position.\par}
{\leftskip .5in 现在这一段,我们将其调整为左边距为半英尺,而
右边距保持原样。\par}

~%{\rightskip .5in Finally, we'll indent the right margin by half
~%an inch and leave the left margin at its usual position.\par}
{\rightskip .5in 最后,我们将本段调整为右边距半英尺,而左边距保持
原样。\par}
~%\bye % end the document
\bye % 结束此文档
:::
~%\xmpheader !xmpnum/{Fonts and special characters}% !xmpheaddef
\xmpheader !xmpnum/{字体和特殊字符}% !xmpheaddef
\chardef \\ = `\\ % Let \\ denote a backslash.
~^^{fonts} ^^{characters//special} ^^{accents}
~^^{music symbols} ^^{card suits}
~^^|$| ^^|&| ^^|#| ^^|_| ^^|%| ^^|^| ^^|~| ^^|{| ^^|}| \indexchar \
~%{\it Here are a few words in an italic font}, {\bf a
~%few words in a boldface font}, {\it and a\/ {\bf mixture}
~%of the two, with two\/ {\rm roman words} inserted}.
~%Where an italic font is followed by a nonitalic font, we've
~%inserted an ``italic correction'' ({\tt \\/}) to make the
~%spacing look right.
{\it 这是一些斜体字词}, {\bf 一些粗体字词},
{\it 以及一些 {\bf 二者混合的效果},
我们在中间插入一些 {\rm 正常的字词} }.
当斜体字后面紧跟非斜体字时, 我们在其间插入一个``倾斜校正''
({\tt \\/}),使得中间的空白看起来较舒适。
~%Here's a {\sevenrm smaller} word---but the standard \TeX\
~%fonts won't give you anything smaller than {\fiverm this}.
这是一些{\sevenrm 更小的} 字词---可是标准的 \TeX\
字体并不会给出小于 {\fiverm 如此大小}的字体。

~%If you need any of the ten characters:
如果你需要以下十个字符任意之一:
\medskip
\centerline{\$ \quad \& \quad \# \quad \_ \quad \% \quad
   \char `\^ \quad \char `\~ \quad $\{$ \quad
   $\}$ \quad $\backslash$}
% The \quad inserts an em space between characters.
\medskip
~%\noindent  you'll need to write them a special way.  Look at
~%the facing page to see how to do it.
\noindent 你需要通过特殊的方式来写他们。参照对开页,
学习正确的记录这些特殊字符的方法.

~%\TeX\ has the accents and letters that you'll need
~%for French words such as {\it r\^ ole\/} and  {\it \'
~%el\` eve\/}, for German words such as {\it Schu\ss\/},
~%and for words in several other languages as well.
~%You'll find a complete list of \TeX's accents and letters
~%of European languages on !xrefdelim[accents] and !xrefdelim[fornlets].
\TeX\ 拥有一些语音和字母,可以用于某些法语词,
例如:{\it r\^ ole\/} 以及{\it \'el\` eve\/},
或者德语词,例如 {\it Schu\ss\/},或者某些别的什么语言。
你可以在 !xrefdelim[accents] 和 !xrefdelim[fornlets] 处
找到 \TeX\ 的 欧洲语音和字母表。

~%You can also get Greek letters such as ``$\alpha$'' and
~%``$\Omega$'' for use in math, card suits such as
~%``$\spadesuit$'' and ``$\diamondsuit$'', music symbols
~%such as ``$\sharp$'' and ``$\flat$'', and many other special
~%symbols that you'll find listed on !xrefdelim[specsyms].
~%\TeX\ will only accept these sorts of special symbols in its
~%``math mode'', so you'll need to enclose them
~%within `{\tt \$}' characters.
你也可以写一些希腊字母,例如数学中的 ``$\alpha$'' 以及
``$\Omega$'' , 扑克牌中的
``$\spadesuit$'' 以及 ``$\diamondsuit$'', 一些音乐符号,例如:
``$\sharp$'' 和 ``$\flat$'', 以及一些你可以在 !xrefdelim[specsyms]
找到的各种符号。\TeX\ 仅在 ``数学模式'' 下接受这些符号, 因此,
你需要将它们放在两个 `{\tt \$}' 字符中间。
~%\bye % end the document
\bye % 结束此文档
:::
~%\xmpheader !xmpnum/{Interline spacing}% !xmpheaddef
\xmpheader !xmpnum/{行间间距}% !xmpheaddef
~^^{spacing//interline} ^^{baselines}
\baselineskip = 2\baselineskip % double spacing
\parskip = \baselineskip % Skip a line between paragraphs.
\parindent = 3em % Increase indentation of paragraphs.

% The following macro definition gives us nice inline
% fractions.  You'll find it in our eplain macros.
\def\frac#1/#2{\leavevmode
   \kern.1em \raise .5ex \hbox{\the\scriptfont0 #1}%
   \kern-.1em $/$%
   \kern-.15em \lower .25ex \hbox{\the\scriptfont0 #2}%
}%

~%Once in a while you may want to print a document with extra
~%space between the lines.  For instance, bills before Congress
~%are printed this way so that the legislators can mark them up.
~%For the same reason, book publishers usually insist that
~%manuscripts be double-spaced. Double spacing is rarely
~%appropriate for finished documents, however.
有的时候,你会想在文档的文字行之间增加一些空白。例如:议会的
某些讨论稿需要这样做,使得议院们能够在其上作一些标记。书商们
也基于同样的理由,要求作者提交的手稿具有双倍的行间距。当然,
对最终的出版物来说,双倍行间距的情形极其少见。

~%A baseline is an imaginary line that acts like the lines
~%on a pad of ruled paper.   You can control the interline
~%spacing---what printers call ``leading''---%
~%by setting the amount of space between baselines. Take a
~%look at the input to see how to do it. You could use
~%the same method for $1\;1/2$ spacing as well, using {\tt 1.5}
~%instead of {\tt 2}. (You can also write $1\frac 1/2$
~%a nicer way.)
基线是一条想象出来的线,其作用类似于带横格标记的纸张上
横格线的作用。你可以通过控制基线间的距离来控制两行文字
之间的距离 --- 打印机管这个距离叫做 ``起始空白''。具体
的做法可以参看这段的源文件。你可以用{\tt 1.5}取代{\tt 2}
获得 $1\;1/2$空白的效果。(或者可以写成更好看的 $1\frac 1/2$
形式。)

% Here we've used the macro definition given above.

~%For this example we've also increased the paragraph indentation
~%and skipped an extra line between paragraphs.
做为举例,我们这里增加了段落的缩进,并且在两段之间增加了一个空行。

~%\bye % end the document
\bye % 结束此文档
:::
~%\xmpheader !xmpnum/{Spacing, rules, and boxes}% !xmpheaddef
\xmpheader !xmpnum/{间隔、标线和盒子}% !xmpheaddef
~^^{description lists} ^^{boxes//drawing} ^^{revision bars}
~%Here's an example of a ``description list''.  In practice you'd
~%be better off using a macro to avoid the repetitive constructs
~%and to make sure that the subhead widths are wide enough:
这里展示的是 ``描述列表''的实例。实际使用时,最好将这些重复结构
定义成宏,并且保证子标题的宽度足够宽。

\bigskip
~% Call the indentation for descriptions \descindent
~% and set it to 8 picas.
% 用 \descindent 表示描述列表的缩进量,并设定它的值为 8 皮卡。
\newdimen\descindent \descindent = 8pc
~% Indent paragraphs by \descindent.
~% Skip an additional half line between paragraphs.
% 整个段落缩进 \descindent。段落间增加半个行距的间距。
{\noindent \leftskip = \descindent  \parskip = .5\baselineskip
~% Move the description to the left of the paragraph.
% 将描述放在段落的左边。
\llap{\hbox to \descindent{\bf Queen of Hearts\hfil}}%
An ill-tempered woman, prone to saying ``Off with his
head!!''\ at the slightest provocation.\par
\noindent\llap{\hbox to \descindent{\bf Cheshire Cat\hfil}}%
A cat with an enormous smile that Alice found
in a tree.\par
\noindent\llap{\hbox to \descindent{\bf Mock Turtle\hfil}}%
A lachrymose creature, quite a  storyteller, who was a
companion to the Gryphon. Reputedly the prin\-cipal ingredient
of Mock Turtle Soup.
\par}
\bigskip\hrule\bigskip % A line with vertical space around it.
~%Here's an example of some words in a ruled box, just as
~%Lewis Carroll wrote them:
这个例子在一个标线盒子中放入一些词句,它们是路易斯·卡罗所写的:
\bigskip
~% Put 8pt of space between the text and the surrounding rules.
% 在文本和四周标线之间留有 8pt 宽的间隔。
\hbox{\vrule\vbox{\hrule
   \hbox spread 8pt{\hfil\vbox spread 8pt{\vfil
~%    \hbox{Who would not give all else for twop}%
~%    \hbox{ennyworth only of Beautiful Soup?}%
      \hbox{谁不是最想尝一尝,}%
      \hbox{两便士一碗的好汤?}%
   \vfil}\hfil}
\hrule}\vrule}%

\bigskip\line{\hfil\hbox to 3in{\leaders\hbox{ * }\hfil}\hfil}
\bigskip

\line{\hskip -4pt\vrule\hfil\vbox{
~%Here we've gotten the effect of a revision bar on the material
~%in this paragraph.  The revision bar might indicate a change.}}
我们给此段落内容添加了修订线效果。修订线用于表示内容改动。}}
~%\bye % end the document
\bye % 结束此文档
:::
~%\xmpheader !xmpnum/{Odds and ends}% !xmpheaddef
\xmpheader !xmpnum/{杂项}% !xmpheaddef
~^^{hyphenation} ^^{theorems} ^^{lemmas} ^^{itemized lists}
~^^{flush left} ^^{flush right} ^^{centering}
~%\chardef \\ = `\\ % Let \\ denote a backslash.
\chardef \\ = `\\ % 用 \\ 表示反斜杠。
\footline{\hfil{\tenit - \folio -}\hfil}
~\global\footline{\hfil{\tenit - \folio\ -}\hfil}
~% \footline provides a footer line.
~% Here it's a centered, italicized page number.
% \footline 给出页脚行;此处为居中的意大利体页码。
~%\TeX\ knows how to hyphenate words, but  it isn't infallible.
~%If you are discussing the chemical
~%${\it 5}$-[p-(Flouro\-sul\-fonyl)ben\-zoyl]-l,%
~%$N^6$-ethe\-no\-adeno\-sine
~%and \TeX\ complains to you about an ``overfull hbox'', try
~%inserting some ``discretionary hyphens''. The notation
~%`{\tt \\-}' tells \TeX\ about a  dis\-cre\-tion\-ary hyphen,
~%that is, one that it might not have inserted otherwise.
\TeX\ 懂得如何将单词连字化,但它并非绝对可靠的。若你在讨论化学
${\it 5}$-[p-(Flouro\-sul\-fonyl)ben\-zoyl]-l,%
$N^6$-ethe\-no\-adeno\-sine 时,
\TeX\ 给出关于``水平盒子溢出''的警告,
你可以尝试插入某些``自定连字符''。
`{\tt \\-}'记号告诉 \TeX\ 一个自定连字符位置;
在此位置 \TeX\ 本来是不会插入连字符的。
\medskip
~%{\raggedright   You can typeset text unjustified, i.e., with
~%an uneven right margin. In the old days, before word
~%processors were common, typewritten documents were
~%unjustified because there was no convenient alternative.
~%Some people  prefer text to be  unjustified so that the
~%spacing between words can be uniform.  Most books are set
~%with justified margins, but not all. \par}
{\raggedright 你可以不对齐地排版文字,即让右页边不对齐。
在过去,文字处理软件还未普及,
因为没有其他方便的选择,排版的文档都是不对齐的。
有些人更喜欢不对齐地排版文字,因为此时单词间距是一致的。
大部分书籍都设置为页边对齐,但并非都是。\par}

~%\proclaim Assertion 27. There is an easy way to typeset
~%the headings of assertions, lemmas, theorems, etc.
\proclaim 断言 27. 有很简单的方法排版断言、引理、定理等的标题。

~%Here's an example of how to typeset an itemized list two
~%levels deep.  If you need more levels, you'll have to
~%program it yourself, alas.
~%\smallskip
~%\item {1.} This is the first item.
~%\item {2.} This is the second item.  It consists of two
~%paragraphs.  We've indented the second paragraph so that
~%you can easily see where it starts.
这个例子显示如何排版两层的有序列表。
如果需要更多层级,唉,你得自己编程。
\smallskip
\item {1.} 这是第一个列表项。
\item {2.} 这是第二个列表项。它由两个段落组成。
为让你看清楚第二个段落从哪里开始,我们缩进了该段落。

~%\item{} \indent The second paragraph has three subitems
~%underneath it.
~%\itemitem {(a)} This is the first subitem.
~%\itemitem {(b)} This is the second subitem.
~%\itemitem {(c)} This is the third subitem.
~%\item {$\bullet$} This is a strange-looking item because it's
~%completely different from the others.
\item{} \indent 第二个段落下边有三个子列表项。
\itemitem {(a)} 这是第一个子列表项。
\itemitem {(b)} 这是第二个子列表项。
\itemitem {(c)} 这是第三个子列表项。
\item {$\bullet$} 这是一个看起来很奇怪的列表项,
因为它和其他列表项不同。
\smallskip
~%\leftline{Here's a left-justified line.$\Leftarrow$}
~%\rightline{$\Rightarrow$Here's a right-justified line.}
~%\centerline{$\Rightarrow$Here's a centered line.$\Leftarrow$}
\leftline{这是一个左对齐的行。$\Leftarrow$}
\rightline{$\Rightarrow$这是一个右对齐的行。}
\centerline{$\Rightarrow$这是一个居中的行。$\Leftarrow$}
~% Don't try to use these commands within a paragraph.
% 不要在段落内部使用这些命令。
~%\bye % end the document
\bye % 结束此文档
:::
~%\xmpheader !xmpnum/{Using fonts from other sources}% !xmpheaddef
\xmpheader !xmpnum/{使用其他来源的字体}% !xmpheaddef
~\xrdef{palatino}
~\idxref{Palatino fonts}
~\idxref{Zapf, Hermann}
~\idxref{Computer Modern fonts} ^^{\Metafont}
\font\tenrm = pplr % Palatino
% Define a macro for invoking Palatino.
\def\pal{\let\rm = \tenrm \baselineskip=12.5pt \rm}
\pal % Use Palatino from now on.

~%You aren't restricted to using the Computer Modern fonts that
~%come with \TeX.  Other fonts are available from many sources,
~%and you may prefer them.  For instance, we've set this page
~%in 10-point Palatino Roman. Palatino was designed by
~%Hermann Zapf, considered to be one of the greatest type
~%designers of the twentieth century.  This page will
~%give you some idea of what it looks like.
你并不局限于只能使用随 \TeX 发行的 Computer Modern 系列字体。
从其他很多渠道你都可以得到一些你可能会更喜欢的字体。例如我们
当前页面就被设置为 10p 大小的 Palatino Roman 字体。Palatino
是一种由 Hermann Zapf 所设计的字体,它被认为是二十世纪最伟大的
字体设计之一。从本页的输出可以获得关于这个字体的一些直观印
象(由于译本的缘故,本页的字体并未被设为 Palatino 字体)。

~%Fonts can be provided either as outlines or as bitmaps.  An
~%outline font describes the shapes of the characters, while a
~%bitmap font specifies each pixel (dot) that makes up each
~%character.  A font outline can be used to generate many
~%different sizes of the same font.  The Metafont program
~%that's associated with \TeX\  provides a particularly
~%powerful way of generating bitmap fonts, but it's not the
~%only way.
字体既可以是轮廓字体,也可以是位图字体。轮廓字体的意思是它
描述的是各个字符的形状,而位图字体则是标识了每个字符形状所
占据的像素点。轮廓字体可用来产生同一种字体的多种不同大小。
随 \TeX\ 而发行的 metafont 程序是一个产生位图字体的强大工具,
而且它并不是唯一的工具。

~%The fact that a single outline can generate a great range of
~%point sizes for a font tempts many vendors of digital
~%typefaces to provide just one set of outlines for a typeface
~%such as Palatino Roman.  This may be a sensible economic
~%decision, but it is an aesthetic sacrifice.  Fonts cannot be
~%scaled up and down linearly without loss of quality.
~%Larger sizes of letters should not, in general, have the
~%same proportions as smaller sizes; they just don't look
~%right.  For example, a font that's linearly scaled down will
~%tend to have too little space between strokes, and its
~%x-height will be too~small. % tie added to avoid widow word
由于单一的轮廓字体文件既可生成多种不同大小的字体,因此很多
数字字体提供商倾向于仅提供字体的单一文件 ——例如 Palatino
Roman 即是如此。这也许是个经济的决定,可它却同时是个美学上
的重大牺牲。字体无法做到在不损害其显示质量的前提下,线性地
放大或者缩小。一个字符的大字体版本,通常来说,与小字体的版
本应该有不同的比例 —— 而如果我们坚持这么做的话,其结果看
起来将不会那么好看。例如:如果将一个字体线性缩小的话,其各
笔画间的间距将会显得过窄,而它的 x-height 值也会显得过小。

~%A type designer can compensate for these changes by
~%providing different outlines for different point sizes, but
~%it's necessary to go to the expense of designing these
~%different outlines.  One of the great advantages of Metafont
~%is that it's possible to parameterize the descriptions of
~%characters in a font.  Metafont can then maintain the
~%typographical quality of characters over a range of point
~%sizes by adjusting the character shapes accordingly.
字体的设计者可以为不同大小的字体设计不同的轮廓形状,从而
对这个问题进行弥补,可这同时需要付出额外的代价。使用 Metafont
程序的一个巨大的好处就是可以将字符符号形状的描述参数化。
这样在改变字体大小的时候,Metafont 可以视情况做一些自动处理,
从而得到较好的效果。
~%\bye % end the document
\bye % 结束此文档
:::
~\idxref{蘑菇}
~%\xmpheader !xmpnum/{A ruled table}% !xmpheaddef
\xmpheader !xmpnum/{标线表格}% !xmpheaddef
\bigskip
~%\offinterlineskip % So the vertical rules are connected.
\offinterlineskip % 让竖直标线连接起来
~% \tablerule constructs a thin rule across the table.
% \tablerule 构造横穿表格的细标线
\def\tablerule{\noalign{\hrule}}
~% \tableskip creates 9pt of space between entries.
% \tableskip 在单元格之间生成 9pt 的空隙
\def\tableskip{\omit&height 9pt&&&\omit\cr}
~% & separates templates for each column. TeX substitutes
~% the text of the entries for #. We must have a strut
~% present in every row of the table; otherwise, the boxes
~% won't butt together properly, and the rules won't join.
% 用 & 分开模版各列。TeX 将把 # 替换为单元格的文本。
% 在表格每行都要有一个支架(strut);
% 否则盒子将不能良好接合,标线也不能连接起来。
~%\halign{\tabskip = .7em plus 1em  % glue between columns
~% Use \vtop for short multiline entries in the first column.
~% Typeset the lines ragged right, without hyphenation.
~%   \vtop{\hsize=6pc\pretolerance = 10000\hbadness = 10000
~%      \normalbaselines\noindent\it#\strut}%
~%  &\vrule #&#\hfil &\vrule #% the rules and middle column
\halign{\tabskip = .7em plus 1em  % 各列之间的粘连
% 用 \vtop 得到表格第一列中的多行单元格。
% 让多行文本右边不对齐,且不连字化。
   \vtop{\hsize=6pc\pretolerance = 10000\hbadness = 10000
      \normalbaselines\noindent\it#\strut}%
  &\vrule #&#\hfil &\vrule #% 标线和中间各列
~%% Use \vtop to get whole paragraphs in the last column.
~%  &\vtop{\hsize=11pc \parindent=0pt \normalbaselineskip=12pt
~%    \normalbaselines \rightskip=3pt plus2em #}\cr
% 用 \vtop 得到表格最后一列中的段落。
  &\vtop{\hsize=11pc \parindent=0pt \normalbaselineskip=12pt
    \normalbaselines \rightskip=3pt plus2em #}\cr
~% The table rows begin here.
% 表格各行从这里开始。
~%\noalign{\hrule height2pt depth2pt \vskip3pt}
~%  % The header row spans all the columns.
~%  \multispan5\bf Some Choice Edible Mushrooms\hfil\strut\cr
\noalign{\hrule height2pt depth2pt \vskip3pt}
  % 表头行横跨各列。
  \multispan5\bf 一些上等的食用蘑菇\hfil\strut\cr
~%\noalign{\vskip3pt} \tablerule
~%  \omit&height 3pt&\omit&&\omit\cr
~%  \bf Botanical&&\bf Common&&\omit \bf Identifying \hfil\cr
~%\noalign{\vskip -2pt}% close up lines of heading
~%  \bf Name&&\bf Name &&\omit \bf Characteristics \hfil\cr
\noalign{\vskip3pt} \tablerule
  \omit&height 3pt&\omit&&\omit\cr
  \bf 学名&&\bf 常用名&&\omit \bf 识别特征 \hfil\cr
~%\tableskip Pleurotus ostreatus&&Oyster mushroom&&
~%  Grows in shelf\kern 1pt like clusters on stumps or logs,
~%  % without the kern, the `f' and `l' would be too close
~%  pink-gray oyster-shaped caps, stem short or absent.\cr
\tableskip Pleurotus ostreatus&&平菇&&
  Grows in shelf\kern 1pt like clusters on stumps or logs,
  % 若不加上 \kern ,`f' 和 `l' 将挨得太近
  pink-gray oyster-shaped caps, stem short or absent.\cr
~%\tableskip Lactarius hygrophoroides&&Milky hygroph&&
~%  Butterscotch-brown cap and stem, copious white latex,
~%  often on ground in woods near streams.\cr
\tableskip Lactarius hygrophoroides&&稀褶乳菇&&
  Butterscotch-brown cap and stem, copious white latex,
  often on ground in woods near streams.\cr
~%\tableskip Morchella esculenta&&White morel&&Conical cap
~%  with black pits and white ridges; no gills. Often found
~%  near old apple trees and dying elms in the spring.\cr
\tableskip Morchella esculenta&&羊肚菌&&Conical cap
  with black pits and white ridges; no gills. Often found
  near old apple trees and dying elms in the spring.\cr
~%\tableskip Boletus edulus&&King bolete&&Reddish-brown to
~%  tan cap with yellow pores (white when young),
~%  bulbous stem, often near conifers, birch, or~aspen.\cr
\tableskip Boletus edulus&&美味牛肝菌&&Reddish-brown to
  tan cap with yellow pores (white when young),
  bulbous stem, often near conifers, birch, or~aspen.\cr
\tableskip \tablerule \noalign{\vskip 2pt} \tablerule
}\bye
:::
~%\xmpheader !xmpnum/{Typesetting mathematics}% !xmpheaddef
\xmpheader !xmpnum/{排版数学公式}% !xmpheaddef
~^^{数学}
~%For a spherical triangle with sides $a$, $b$, and $c$, and
~%opposite angles $\alpha$, $\beta$, and $\gamma$, we have:
设球面三角形的三条边长度分别为$a$,$b$ 和 $c$,
它们的对角分别为 $\alpha$,$\beta$ 和 $\gamma$,我们有:
~%$$\cos \alpha = -\cos \beta \cos \gamma +
~%  \sin \beta \sin \gamma \cos \alpha \quad
~%  \hbox{(Law of Cosines)}$$
$$\cos \alpha = -\cos \beta \cos \gamma +
  \sin \beta \sin \gamma \cos \alpha \quad
  \hbox{(余弦定理)}$$
~%and:
且有:
~%$$\tan {\alpha \over 2} = \sqrt{
~%  {- \cos \sigma \cdot \cos(\sigma - \alpha)} \over
~%  {\cos (\sigma - \beta) \cdot \cos (\sigma - \gamma)}},\quad
~%  \hbox{where $\sigma = {1 \over 2}(a+b+c)$}$$
$$\tan {\alpha \over 2} = \sqrt{
  {- \cos \sigma \cdot \cos(\sigma - \alpha)} \over
  {\cos(\sigma - \beta) \cdot \cos(\sigma - \gamma)}},\quad
  \hbox{其中 $\sigma = {1 \over 2}(a+b+c)$}$$
~%We also have:$$\sin x = {{e^{ix}-e^{-ix}}\over 2i}$$
~%and:
~%$$\int _0 ^\infty {{\sin ax \sin bx}\over{x^2}}\,dx
~%% The \, above produces a thin space
~%  = {\pi a\over 2}, \quad \hbox{if $a < b$}$$
我们还有:$$\sin x = {{e^{ix}-e^{-ix}}\over 2i}$$
且有:
$$\int _0 ^\infty {{\sin ax \sin bx}\over{x^2}}\,dx
% 上面的 \, 生成一个小间隔
  = {\pi a\over 2}, \quad \hbox{如果 $a < b$}$$

~%\noindent The number of combinations ${}_nC_r$ of $n$
~%things taken $r$ at a time is:
\noindent 从 $n$ 个物品中任取 $r$ 个的组合数 ${}_nC_r$ 为:
$$C(n,r) = {}_nC_r = {n \choose r} =
  {{n(n-1) \cdots (n-r+1)} \over {r(r-1) \cdots (1)}} =
  {{n!!}\over {r!!(n-r)!!}}$$

\noindent
~%The value of the determinant $D$ of order $n$:
$n$ 阶行列式 $D$ 的值:
$$D = \left|\matrix{a_{11}&a_{12}&\ldots&a_{1n}\cr
  a_{21}&a_{22}&\ldots&a_{2n}\cr
  \vdots&\vdots&\ddots&\vdots\cr
  a_{n1}&a_{n2}&\ldots&a_{nn}\cr}\right| $$
~%is defined as the sum of $n!!$ terms:
定义为下面$n!!$项的和:
$$\sum\>(\pm)\>a_{1i}a_{2j} \ldots a_{nk}$$
~% The \> above produces a medium space.
~%where $i$, $j$, \dots,~$k$\/ take on all possible values
~%between $1$ and $n$, and the sign of the product is
~%$+$ if the sequence $i$, $j$, \dots,~$k$\/ is an
~%even permutation and $-$ otherwise.  Moreover:
其中 $i$, $j$, \dots,~$k$\/ 取遍 $1$ 到 $n$ 的所有可能值,
且当排列 $i$, $j$, \dots,~$k$\/ 为偶置换时乘积取 $+$ 号,
否则取 $-$ 号。此外:
$$Q(\xi) = \lambda_1 y_1^2 \sum_{i=2}^n \sum_{j=2}^n y_i
b_{ij} y_j,\qquad B = \Vert b_{ij} \Vert = B'$$
\bye
:::
~%\xmpheader !xmpnum/{More mathematics}% !xmpheaddef
\xmpheader !xmpnum/{更多数学内容}% !xmpheaddef
%^^{math}
^^{数学}
~%The absolute value of $X$, $|x|$, is defined by:
~%$$|x| = \cases{x, &if $x\ge 0$;\cr
~%-x,&otherwise.\cr}$$
~%Now for some numbered equations.
~%It is the case that for $k \ge 0$:
~%$$x^{k^2}=\overbrace{x\>x\>\cdots\> x}^{2k\ \rm times}
~%\eqno (1)$$
$X$ 的绝对值 $|x|$ 定义为::
$$|x| = \cases{x, &if $x\ge 0$;\cr
-x,&otherwise.\cr}$$
现在写一些标号了的公式。
在 $k \ge 0$ 时,我们有:
$$x^{k^2}=\overbrace{x\>x\>\cdots\> x}^{2k\ \rm times}
\eqno (1)$$

~%Here's an example that shows some spacing controls, with
~%a number on the left:
这是一个展示空格间距调整和公式左编号的例子:
~%$$[u]!negthin[v][w]\,[x]\>[y]\;[z]\leqno(2a)$$
$$[u]!negthin[v][w]\,[x]\>[y]\;[z]\leqno(2a)$$
~%The amount of space between the items in brackets
~%gradually increases from left to right. (We've made
~%the space between the first two items be {\it less\/}
~%than the natural space.)
从做到右公式中的项和括号的距离逐步增大。
(我们把前两项的距离设定得比正常距离 {\it 少\/}。
~%It is sometimes the case that $$\leqalignno{
~%u'_1 + tu''_2 &= u'_2 + tu''_1&(2b)\cr
~%\hat\imath &\ne \hat \jmath&(2c)\cr
~%\vec {\vphantom{b}a}&\approx \vec b\cr}$$
~%% The \vphantom is an invisible rule as tall as a `b'.
有时候, $$\leqalignno{
u'_1 + tu''_2 &= u'_2 + tu''_1&(2b)\cr
\hat\imath &\ne \hat \jmath&(2c)\cr
\vec {\vphantom{b}a}&\approx \vec b\cr}$$
% \vphantom 是一个不可见的和“b”一样高的盒子。
~%The result is of order $O(n \log\log n)$. Thus
~%$$\sum_{i=1}^n x_i = x_1+x_2+\cdots+x_n
~%= {\rm Sum}(x_1,x_2,\ldots,x_n). \eqno(3)$$
结果为 $O(n \log\log n)$. 因此
$$\sum_{i=1}^n x_i = x_1+x_2+\cdots+x_n
= {\rm Sum}(x_1,x_2,\ldots,x_n). \eqno(3)$$
~%and
~%$$dx\,dy = r\,dr\,d\theta!negthin.\eqno(4)$$
~%The set of all $q$ such that $q\le0$ is written as:
~%$$\{\,q\mid q\le0\, \}$$
且
$$dx\,dy = r\,dr\,d\theta!negthin.\eqno(4)$$
满足 $q\le0$ 的所有的 $q$ 可以被写作:
$$\{\,q\mid q\le0\, \}$$
~%Thus
~%$$\forall x\exists y\;P(x,y)\Rightarrow
~%\exists x\exists y\;P(x,y)$$
~%where
~%$$P(x,y) \buildrel \rm def \over \equiv
~%\hbox{\rm any predicate in $x$ and $y$} . $$
因此
$$\forall x\exists y\;P(x,y)\Rightarrow
\exists x\exists y\;P(x,y)$$
其中
$$P(x,y) \buildrel \rm def \over \equiv
\hbox{\rm 任何一个能够得出 $x$ 和 $y$} . $$
\bye
:::
