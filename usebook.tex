% -*- coding: utf-8 -*-
% This is part of the book TeX for the Impatient.
% Copyright (C) 2003 Paul W. Abrahams, Kathryn A. Hargreaves, Karl Berry.
% See file fdl.tex for copying conditions.

\input macros
%\chapter{Using this book}
\chapter{使用本书}

\chapterdef{usebook}

%This book is a do-it-yourself guide and handbook for \TeX.
%Here in this section we tell you how to use the book to maximum advantage.

本书是一本介绍 \TeX{} 的自学手册。
在本章中, 我们将指导你如何更好地使用本书。

% We recommend that you first either read or skim in sequence Sections
% \chapternum{usebook} through \chapternum{examples},
% which tell you what you need to know in order to get started using \TeX.
% If you've already had experience using \TeX, it will still be helpful
% to know what kinds of information are in these sections of the book.
% Sections~\chapternum{concepts}--\chapternum{tips}, which
% occupy most of
% the rest of the book, are designed to be accessed randomly.
% Nevertheless, if you're the kind of person who likes to read reference manuals,
% you'll find that it \emph{is} possible to proceed sequentially if you're
% willing to take a lot of detours at first.

我们建议你先阅读或浏览 \chapternum{usebook} 到 \chapternum{examples}章,
这些地方会告诉你如何上手 \TeX。
如果你已经使用过 \TeX, 这些地方仍会值得一看,
至少你可以知道这些章节介绍了些什么内容。
%本书的重头戏是\chapternum{concepts}--\chapternum{tips}章,
%这些章节可以任意选择地进行阅读.
其余部分,主要是\chapternum{concepts}--\chapternum{tips}章,
可以有选择地翻阅。
当然, 如果你是一个很喜爱阅读参考手册的人,
你会发现从头到尾地把这本书阅读一遍\emph{是}完全可行的,
当然你会为此在开头学习时绕些弯路。

% In \chapterref{usingtex}, ``Using \TeX'',
% we explain how to produce a \TeX\ document from a
% \TeX{} input file.
% We also describe the conventions for preparing that input file,
% explain a little about how \TeX{} works, and tell you about additional
% resources that are available.
% Reading this section will help you understand
% the examples in the next section.

在\chapterref{usingtex},``使用 \TeX'',
我们解释如何把一个 \TeX{} 输入文件转为一个 \TeX{} 文档。
%我们还讲述了准备这个输入文件的惯例,
我们还讲述了准备输入文件的一些习惯,
简要地解释了 \TeX{} 是如何工作的, 告诉你可以获取的其它资源信息。
阅读了这个章节, 会有助于你理解下一个章节的例子。

% \chapterref{examples}, ``Examples'',
% contains a sequence of
% examples that illustrate the use of \TeX.
% Each example consists of a page of output
% together with the input that we used to create it.
% These examples will orient you and help you
% locate the more detailed material that you'll need as you go.
% By seeing which commands are used in the input, you'll know
% where to look for more detailed information on how to achieve
% the effects shown in the output.
% The
% examples can also serve as models for simple documents, although we must
% caution you that
% because we've tried to pack a variety of \TeX\ commands into a small
% number of pages,
% the examples are not necessarily illustrations of good or complete
% document~design.

\chapterref{examples}, ``例子'',
包括了一系列的例子来描述 \TeX{} 的功用.
每个例子包括了一页的输出内容, 也包括了输出该页内容所输入的源代码.
这些例子会指导你且帮助你找到你所需要的更详细的资料.
通过观察输入使用了什么命令,
你会知道在哪里能够找到输出效果所使用方法的更详细的信息.
这些例子也可以作为简单的文档的模版, 当然我们必需提醒你,
因为我们尝试把很多的 \TeX{} 命令放在这几页纸中,
这些例子不可能描述好的或者完整的文档设计.

% As you read the explanation of a command, you may encounter
% some unfamiliar technical terms.
% In \chapterref{concepts}, ``Concepts'',
% we define and explain these terms.
% We also discuss other topics that aren't covered
% elsewhere in the book.
% The inside back cover of the book contains a list of all the
% concepts and the pages where they are described.
% We suggest that you make a copy of this list and keep it nearby
% so that you'll be able to identify and look up an unfamiliar
% concept immediately.

当你看这些命令的解释时, 你会遇到很多不熟悉的技术术语.
在 \chapterref{concepts}, ``概念''中, 我们定义并解释了这些术语.
我们同时还讨论了本书其它地方没有涉及到的话题.
本书书背内页列出了所有的概念名称以及它们所出现的位置.
我们建议你把书背内页复印下来, 放在旁边,
这样你就能在遇到不熟悉的概念时马上查找到它们.

% \TeX's commands are its primary vocabulary,
% and the largest part of this book is
% devoted to explaining them.  In Sections~\chapternum{paras}
% through~\chapternum{general} we describe the commands.
% You'll find general information about the command descriptions
% on \xrefpg{cmddesc}.
% The command descriptions are arranged
% functionally, rather like a thesaurus, so if you know what you want to
% do but you don't know which command does it for you, you can use the
% table of contents to guide you to the right group of commands.
% Commands that we think are both particularly useful and easy to understand
% are indicated with a pointing~hand~(\hand).

\TeX{} 的命令是它的主要词汇, 本书最主要的部分就是用来解释这些命令.
从第~\chapternum{paras}~章到第~\chapternum{general}~章, 我们解释了这些命令.
你可以在\xrefpg{cmddesc}找到命令描述的描述惯例.
这些命令根据功能分门别类, 而不是按字典顺序,
所以如果你知道你想做什么, 但不知那该用什么命令,
你可以使用目录来引导你找到正确的命令组.
我们给我们认为常用且容易理解的命令标上了一个手形符号~(\hand).

%\chapterref{capsule}, ``Capsule summary of commands'', is a
%specialized index that complements the more complete descriptions
%in Sections~\chapternum{paras}--\chapternum{general}.
%It lists \TeX's commands
%alphabetically, with a brief explanation of each command
%and a reference to the page
%where it is described more completely.  The capsule summary
%will help you when you just want a quick reminder of what a command
%does.
\chapterref{capsule}为``命令速查表'',这个专门的索引与%
第~\chapternum{paras}--\chapternum{general}~章的更完整描述相辅相成。
在这一章中按字母顺序列出了各个 \TeX\ 命令及其简要解释,
并给出该命令的详细描述所在的页码。
此速查表可以让你快速知道某个命令的作用。

%\TeX{} is a complex program that occasionally works its will in
%mysterious ways.
%In \chapterref{tips}, ``Tips and techniques'',
%we provide advice on solving a variety of specific
%problems that you may encounter from time to time.
%And if you're stumped by
%\TeX's error messages, you'll find succor in \chapterref{errors},
%``Making sense of error messages''.
\TeX{} 是个复杂的程序,偶尔地会以诡异的方式工作。
在\chapterref{tips},``建议和技巧''中,
对一系列你时不时会遇到的特定的问题,我们提供了一些建议。
而若你被 \TeX\ 的错误信息难住了,
\chapterref{errors},``理解错误信息''将给你提供帮助。

%The gray tabs on the side of the book will help you locate parts of the
%book quickly.  They divide the book into the following major parts:
%\olist
%\li general explanations and examples
%\li concepts
%\li descriptions of commands (five shorter tabs)
%\li advice, error messages, and the |eplain.tex| macros
%\li capsule summary of commands
%\li index
%\endolist
页边的灰色标签可以帮你快速定位到本书各个部分。
它们将本书划分为下面几个主要部分:
\olist
\li 总体的说明和例子
\li 概念
\li 命令描述(五个短标签)
\li 建议,错误信息和 |eplain.tex| 宏集
\li 命令速查表
\li 索引
\endolist

%In many places we have provided page references to
%^{\texbook} (see \xrefpg{resources} for a citation).
%These references apply to the seventeenth edition of \texbook.
%For other editions, some references may be off by a page or two.
在很多地方我们都给出了到 \texbook\ \mcidxref{texbook} 的页码索引
(要引用该书可以见\xrefpg{resources})。
这些索引适用于\texbook\ 第十七次印刷版。
对于其它版本,有的索引可能相差一两页。


%\section Syntactic conventions
\section 语法约定

%In any book about preparing input for a computer,
%it's necessary to indicate clearly the literal characters that should be typed
%and to distinguish those characters from the explanatory text.
%We use the Computer Modern typewriter font for {\tt literal input
%like this}, and also for the names of \TeX{} commands.
%When there's any possibility of confusion, we enclose \TeX{}
%input in single quotation marks, `{\tt like this}'.
%However, we occasionally use parentheses when we're indicating single
%characters such as (|`|) (you can see why).
在任何涉及到为计算机准备输入的书籍中,
都得明确区分哪些字符是需要逐字输入的,而哪些字符属于解释性文本。
我们用计算机现代打字机字体显示文字输入,比如{\tt literal input},
以及\TeX\ 命令名。在可能导致混淆的地方,
我们将 \TeX\ 命令用单引号围起来,比如`{\tt like this}'。
然而,在表示单个字符时,我们偶尔也会用括号将它围起来,
比如(|`|)(其原因你容易明白)。

%For the sake of your eyes we usually just put spaces
%where you should put spaces. In some places where
%we need to emphasize the space, however,
%we use a `\visiblespace' character
%{\catcode `\ =\other\pix^^| |}%
%to indicate it.
%Naturally enough, this character is called a \emph{visible space}.
%\pix^^{spaces//visible}
通常我们按照你输入空格的方式显示空格,以免让你看花眼。
但为了强调某些空格,我们用`\visiblespace'字符表示它。
\pix\chidxref{\ }
这个字符很自然地被称为\emph{可见空格}.
\pix^^{空格//可见空格}


%\section Descriptions of the commands
\section 命令描述

\xrdef{cmddesc}
%Sections~\chapternum{paras}--\chapternum{general} contain
%a description of what nearly every \TeX{} command does.  ^^{commands}
%Both the primitive commands ^^{primitive//command}
%and those of ^{\plainTeX} are covered.
%The primitive commands are those built into the \TeX{} computer program, while
%the \plainTeX{} commands are defined in a standard file of
%auxiliary definitions (see \xref\plainTeX).
%The only commands we've omitted are those that are used purely locally
%in the definition of \plainTeX\ (\knuth{Appendix~B}).
%The commands are organized as follows:
第~\chapternum{paras}--\chapternum{general}~章描述了几乎每个\TeX\ 命令,
^^{命令}包括原始命令^^{原始的//原始命令}和 \plainTeX\ \mcidxref{plainTeX} 命令。
原始命令是那些内建于 \TeX\ 程序中的命令,
而 \plainTeX{} 命令是那些在标准辅助定义文件中定义的命令(见\xref\plainTeX )。
我们仅仅忽略那些只在\plainTeX\ (\knuth{附录~B})定义中用到的局部命令。
这些命令分为如下几部分:
%\ulist\compact
%\li ``Commands for composing paragraphs'', \chapterref{paras},
%deal with characters, words, lines, and entire paragraphs.
%\li ``Commands for composing pages'', \chapterref{pages},
%deal with pages, their components, and the output routine.
%\li ``Commands for horizontal and vertical modes'', \chapterref{hvmodes},
%have corresponding or identical
%forms for both horizontal modes (paragraphs and hboxes) and vertical
%modes (pages and vboxes).
%These commands provide boxes, spaces, rules, leaders,
%and alignments.
%\li ``Commands for composing math formulas'', \chapterref{math},
%provide capabilities for constructing math formulas.
%\li ``Commands for general operations'', \chapterref{general},
%provide
%\TeX's programming features and
%everything else that doesn't fit into any of the other sections.
%\endulist
\ulist\compact
\li \chapterref{paras},``组段命令'',介绍字符、单词、文本行和整个段落。
\li \chapterref{pages},``组页命令'',介绍页面及其组成部分,以及输出例行程序。
\li \chapterref{hvmodes},``水平和竖直模式命令'',
介绍水平模式(段落和水平盒子)和竖直模式(页面和竖直盒子)中对应或等同的命令。
这些命令提供了盒子、间隔、标线、指引线和对齐。
\li \chapterref{math},``数学公式命令'',介绍构造数学公式的命令。
\li \chapterref{general},``一般操作命令'',
介绍 \TeX\ 的编程功能以及不适合放在其他部分的命令。
\endulist
%You should think of these categories as being suggestive rather than
%rigorous, because the commands don't really fit neatly into these
%(or any other) categories.
你应该将这些分类视为提示性的而不是严格的,
因为把命令都归入这些(或者其他)类别并不完全合适。

%Within each section, the descriptions of the commands are organized
%by function.  When several commands are closely related, they are described as
%a group; otherwise, each command has its own explanation.
%The description of each command
%includes one or more examples and the output
%produced by each example when examples are appropriate (for
%some commands they aren't).
%When you are looking at a subsection containing functionally related
%commands, be sure to check the end of a subsection for a ``see also''
%item that refers you to related commands that are described elsewhere.
各章的命令描述按照其功能组织。
当几个命令密切相关时,它们合起来介绍;否则,每个命令分别解释。
命令的描述中包含一个或多个例子,可能还有例子的输出%
(对有些例子这并不可行)。
若某节内容涉及到功能上相关的其他命令时,
务必看看该节末尾的``参见''项,
那里给出了其它地方介绍到的相关命令所在的页码。

%Some commands are closely related to certain concepts.
%For instance, the |\halign| and |\valign| commands are related to
%``alignment'', the |\def|
%command is related to ``macro'',
%and the |\hbox| and |\vbox| commands are related to ``box''.
%In these cases we've usually given a bare-bones des\-crip\-tion of the
%commands themselves and explained  the underlying ideas
%in the concept.
有些命令与特定的概念密切相关。
例如,|\halign| 和 |\valign| 命令与``对齐''的概念相关,
|\def| 命令与``宏''相关,
而 |\hbox| 和 |\vbox| 命令与``盒子''相关。
对此种情形,在命令这里我们通常只给出粗略介绍,
而将对基本原理的解释放在相关概念中。

%The examples associated with the commands have been typeset with
%^|\parindent|, the paragraph indentation, set to zero so that
%paragraphs are normally unindented.
%This convention makes the examples easier to read.
%In those examples where the paragraph indentation is essential,
%we've set it explicitly to a nonzero value.
命令相关的例子在排版时设定了段落缩进 ^|\parindent| 为零,
因而其中段落一般是不缩进的;这使得例子更加容易阅读。
对有些例子段落缩进是必不可少的,我们显式设置缩进为非零值。

%The pointing hand in front of a command or a group of commands indicates
%that we judged this command or group of commands to be particularly useful
%and easy to understand.
在一个或一组命令前面的手形符号,
表示我们认为这个或这组命令是常用且容易理解的。

%Many commands expect ^{arguments} of one kind or another
%(\xref{arg1}).  The arguments of
%a command give \TeX{} additional information that it needs in order to
%carry out the command.  Each argument is indicated by an italicized
%term in angle brackets that indicates what kind of argument it~is:
某些命令需要特定类型的参量(\xref{arg1})。
命令的参量提供 \TeX{} 执行该命令时所需的额外信息。
每个参量都放在尖括号中,并用意大利体标明它的类型:

%\display{%
%\halign{\<#>\quad&#\hfil\cr
%argument&a single token or some text enclosed in braces\cr
%charcode&a character code, i.e., an integer between $0$ and $255$\cr
%dimen&a dimension, i.e., a length\cr
%glue&glue (with optional stretch and shrink)\cr
%number&an optionally signed integer (whole number)\cr
%register&a register number between $0$ and $255$\cr
%}}
%^^{\<dimen>}
%^^{\<argument>}
%^^{\<charcode>}
%^^{\<glue>}
%^^{\<number>}
%^^{\<register>}
\display{%
\halign{\<#>\quad&#\hfil\cr
argument&单个记号或者放在花括号中的一些文本\cr
charcode&字符码,即在$0$到$255$之间的整数\cr
dimen&尺寸,即长度\cr
glue&粘连(有可选的伸展值和收缩值)\cr
number&可带符号的整型值(整数)\cr
register&寄存器编号,在$0$到$255$之间\cr
}}
\aridxref{dimen}
\aridxref{argument}
\aridxref{charcode}
\aridxref{glue}
\aridxref{number}
\aridxref{register}

% \noindent
% All of these terms are explained in more detail in \chapterref{concepts}.
% In addition, we sometimes use terms such as \<token list> that are either
% self-explanatory or explained in the description of the command.
% Some commands have special formats that require either braces or
% particular words.
% These are set in the same bold font that we use
% for the command headings.

\noindent
所有这些术语会在\chapterref{concepts}中进行详述.
除了这些以外, 我们会使用一些不言自明的或者在后文进行讲述的术语, 比如 \<token list>.
有些命令有自己独特的格式, 比如需要括孤或者特定的字词.
它们会使用和命令标题相同的黑体标出.

% Some commands are parameters (\xref{introparms}) or table entries.
% ^^{parameters//as commands}
% This is indicated in the command's listing.
% You can either use a parameter as an argument or assign a value to it.
% The same holds for table entries.
% We use the term ``parameter'' to refer to entities such as |\pageno|
% that are actually registers but behave just like parameters.
% ^^{registers//parameters as}

有些命令是参数 (\xref{introparms}) 或表格项.
^^{参数//作为命令}
这会在命令的列表中指示出来.
你可以将这个参数作为一个命令的参量,也可以给它赋值。
表格项也是如此。
我们使用``参数''这个术语来指代这些条目, 比如本是寄存器但可当参数使用的 |\pageno|.
^^{寄存器//作为参数}

\ifcompletebook\else
\vskip4em{\sectionfonts\leftline{本章索引}}
\readindexfile{i}
\fi

\endchapter
\byebye
