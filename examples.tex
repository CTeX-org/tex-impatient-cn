% -*- coding: utf-8 -*-
% This is part of the book TeX for the Impatient.
% Copyright (C) 2003 Paul W. Abrahams, Kathryn A. Hargreaves, Karl Berry.
% See file fdl.tex for copying conditions.

\input macros
\chapter{例子}

\chapterdef{examples}


这一章中包括了一组例子, 来帮助你熟悉 \TeX, 同时这些例子还展示了如何使用 \TeX\ 来完成各种排版工作.
每个例子都有一个放在左页的 \TeX\ 排版的结果和放在右页的相对应的 \TeX\ 输入文本.
你可以把这些例子作为模仿的样式, 也可以用来找到你想要的效果的实现命令.
不过要注意的是, 这些例子仅能展示 \TeX\ $900$ 条左右的命令的一小部分.

这里的某些例子是其义自现的——也就是说, 它们在介绍每个所排印出来的功能.
这个介绍很粗略, 因为没有足够的篇幅来讲术所有你想得到的信息.
一个命令的速查摘要 (\chapterref{capsule}) 和索引可以帮你来找到例子中的每个 \TeX\ 功能.

因为我们在设计这些例子时, 把很多的东西放在一起描述,
因此, 这些例子展示了很多的排版效果.
这些例子一般并\emph{不}是好的排版实践模版.
比如例~8 把有些公式编号放在左边, 又把另一些放在了右边.
你永远不会在一个实际的科学出版物中使用这样的公式编号.

\xrdef{xmphead}
除了第一个例子以外, 每个例子都由一个叫 |\xmpheader| 的宏开始 (见\xref{宏}).
我们这样做是为了节省输入文本的篇幅,
否则每个例子开头你都会看到几行你先前已经看到的内容.
|\xmpheader| 会排印出例子的标题和标题后的空白.
你可以参见没有使用 |\xmpheader| 的第一个例子是如何实现这一点的,
然后你就能模仿它了.
除了 |\xmpheader|, 在这里使用的每个命令都是在 \PlainTeX\ 中定义过的.

% The first example does the necessary eject here.
{%
   \let\bye = \relax % We don't want to obey \bye in the example input.
   % These switches can't be done by a macro since \bye is outer.
   \doexamples {xmptext}% Typeset the actual examples.
}%


\endchapter
\byebye
